\documentclass{article}
\usepackage{amsmath, amssymb, amsthm, amsfonts, bm}
\theoremstyle{remark}
\newtheorem*{theorem}{Theorem}
\newtheorem*{remark}{Remark}
\newtheorem*{definition}{Definition}
\newtheorem*{hypothesis}{Hypothesis}
\newtheorem*{corollary}{Corollary}
\theoremstyle{remark}

\usepackage{physics}
\usepackage[a4paper, total={6in,10in}]{geometry}
\usepackage[dvipsnames]{xcolor}
\usepackage{hyperref}
    \hypersetup{colorlinks=true, linkcolor=ForestGreen}
\usepackage{graphicx}
    \graphicspath{{./img/}}
\usepackage{tikz}
\usepackage{ragged2e}

\usepackage{cellspace} 
\newcolumntype{L}[1]{>{\centering\arraybackslash}m{#1}}
\usepackage{makecell}
\setlength{\cellspacetoplimit}{6pt}
\setlength{\cellspacebottomlimit}{6pt}

\usepackage{soul}

\everymath{\displaystyle}

\newcommand{\where}[1]{\begin{flushright}where #1.\end{flushright}}
\newcommand{\wher}[1]{\begin{flushright}#1.\end{flushright}}
\newcommand{\mylabel}[2]{\hyperref[#1]{#2}\label{back:#1}}
\newcommand{\myref}[1]{\hyperref[back:#1]{$\bigstar$}\label{#1}}
\newcommand{\e}{\hat{\vb{e}}}  % unit vector

\begin{document}

    \title{Math Cheatsheet}
    \author{Sol}
    \maketitle

    \section*{The Fourier Transform}
    \begin{enumerate}
        \item $\tilde{f}(k) = \frac{1}{\sqrt{2\pi}}\int_{-\infty}^{\infty} e^{-ikx}f(x) \dd x = \mathcal{F}[f](k)$
                \begin{itemize}
                    \item Real and even ($f^*(x)=f(x) \qand f(x)=f(-x)$)$\implies \tilde{f}^*(k)=\tilde{f}(k) $ ($\tilde{f}$ is \textbf{real})
                    \item Real and odd $\implies$ $\tilde{f}$ is imaginary ($\tilde{f}^*(k)=-\tilde{f}(k) $)
                    \item Linearity, $\mathcal{F}[\alpha f(x)+\beta g(x)]=\alpha\mathcal{F}[f(x)]+\beta\mathcal{F}[g(x)] $
                    \item Rescaling, $\mathcal{F}[f(\alpha x)] = \frac{1}{|\alpha|}\tilde{f}\left(\frac{k}{\alpha}\right) $
                    \item Translation, $\mathcal{F}[f(x-a)]=e^{-ika}\mathcal{F}[f(x)] $
                    \item Exponential, $\mathcal{F}[e^{iax}f(x)](k)=\mathcal{F}[f](k-a) $
                    \item Duality, $\mathcal{F}[\tilde{f}]=f(-k) $
                    \item For \textbf{real} $k$, $\mathcal{F}[f^*](k)=\mathcal{F}[f](-k) $
                    \item Symmetry, $f(-x)=\pm f(x)\implies\tilde{f}(-k)=\pm\tilde{f}(k) $
                    \item Differentiation, $\mathcal{F}\left[\dv{f}{x}\right]=ik\tilde{f}(k) $ ($\mathcal{I}[k\tilde{f}]=-i\dv{f}{x} $)
                    \item $\mathcal{I}\left[\dv{\tilde{f}}{k}\right]=-ixf(x) $ ($\mathcal{F}[xf]=i\dv{\tilde{f}}{k} $)
                    \item Convolution, $\mathcal{F}[f*g]=\sqrt{2\pi}\mathcal{F}[f]\mathcal{F}[g] $
                    \item $\mathcal{F}[fg]=\frac{1}{\sqrt{2\pi}}\mathcal{F}[f]*\mathcal{F}[g] $
                    \item Correlation, $\mathcal{F}[f\otimes g](x)=\sqrt{2\pi}\mathcal{F}[f]^*\mathcal{F}[g] $ (Wiener-Khinchin if $g=f$)
                    \item (TODO: prove the inverse)
                    \item \textbf{Autoconvolution} is $\sqrt{2\pi}\tilde[f]^2$, \textbf{autocorrelation} is $\sqrt{2\pi}|f|^2$
                    \item Parseval's theorem $\int_{-\infty}^{\infty} |f(x)|^2 \dd x = \int_{-\infty}^{\infty} |\tilde{f}(k)|^2 \dd k$ (inverse of Wiener-Khinchin/delta function)
                \end{itemize}
        \item $f(x)=\frac{1}{\sqrt{2\pi}}\int_{-\infty}^{\infty} e^{ikx}\tilde{f}(k) \dd k =\mathcal{I}[\tilde{f}](x)$
        \item Convolution is $(f*g)(x)=\int_{-\infty}^{\infty} f(y)g(x-y) \dd y $
        \item Correlation is $(f\otimes g)\int_{-\infty}^{\infty}[f(y)]^*g(x+y)\dd y $\begin{itemize}
            \item $f(x)\otimes g(x)=f(-x)^* * g(x) $
            \item $f(x)\otimes g(x) = g(-x)^*\otimes f(-x)^* $
            \item If $f$ is hermitian, $f\otimes g=f*g $
            \item If $f,g$ are hermitian, $f\otimes g=g\otimes f $
            \item $(f\otimes g)\otimes(f\otimes g)=(f\otimes f)\otimes(g\otimes g) $
        \end{itemize}
        \item $\delta(t-t_0)=\frac{1}{2\pi}\int_{-\infty}^\infty e^{-i\omega (t-t_0)}\dd \omega$ (prove \textcolor{red}{$\frac{1}{2\pi}$} by transforming twice)
        \item Laplace Transform $\mathcal{L}(f)(s)=\int_{0}^{\infty} f(t)e^{-st} \dd t $
    \end{enumerate}

    \begin{center}
        \begin{tabular}{|c|c|}
            \hline
            $f$ & $\tilde{f}$\\
            \hline
            $e^{-b|x|},\ b>0$ & $\frac{1}{\sqrt{2\pi}}\frac{2b}{k^2+b^2}$\\
            \hline
            $\frac{1}{x^2+b^2} $ & $\sqrt{\frac{\pi}{2b^2}}e^{-b|k|} $\\
            \hline
            $\frac{1}{\sqrt{2\pi\epsilon^2}\exp\left(-\frac{x^2}{2\epsilon^2}\right)}$ & $\frac{1}{\sqrt{2\pi}}exp\left(-\frac{\epsilon^2k^2}{2}\right) $\\
            \hline
            $\delta(x-a) $ & $\frac{1}{\sqrt{2\pi}}e^{-ika} $\\
            \hline
            $H(x-a)e^{-\epsilon(x-a)} $ & $\frac{1}{\sqrt{2\pi}}\frac{e^{-ika}}{\epsilon+ik} $\\
            \hline
            $H(x-a) $ & $\frac{e^{-ika}}{ik\sqrt{2\pi}} $\\
            \hline
            $H(x+a)H(a-x) $ (tophat) & $\sqrt{\frac{2}{\pi}}\frac{\sin (ak)}{k} $\\
            \hline

        \end{tabular}
    \end{center}

    \section*{Vector Calculus}
    Practice List\begin{enumerate}
        \item $(\vb{a}\times\vb{b})_i=\epsilon_{ijk}a_j b_k $
        \item $\det A=\epsilon_{ijk}A_{1i}A_{2j}A_{3k} $
        \item $\det(\vb{e}_i\ \vb{e}_j\ \vb{e}_k)=\epsilon_{ijk} $
        \item $\epsilon_{ijk}\epsilon_{lmn} = \det(\vb{e}_i\ \vb{e}_j\ \vb{e}_k)^T\det(\vb{e}_l\ \vb{e}_m\ \vb{e}_n) = \begin{vmatrix}
            \delta_{il} & \delta_{im} & \delta_{in}\\
            \delta_{jl} & \delta_{jm} & \delta_{jn}\\
            \delta_{kl} & \delta_{km} & \delta_{kn}
        \end{vmatrix}$
        \item Bear in mind that $\boxed{\delta_{kk}=\textcolor{red}{3}=\delta_{k1}\delta_{k1}+\delta_{k2}\delta_{k2}+\delta_{k3}\delta_{k3}}$ ($\delta_{11}=1$) ($\delta_{lm}=\delta_{l1}\delta_{m1}+\delta_{l2}\delta_{m2}+\delta_{l3}\delta_{m3}$)
        \item $\epsilon_{ijk}\epsilon_{imn}=\delta_{jm}\delta_{kn}-\delta_{jn}\delta_{km} $
        \item $\boxed{\epsilon_{ijk}\epsilon_{ijk}=6}$
        \item $(\vb{a}\times\vb{b})\cdot(\vb{c}\times\vb{d})=(\vb{a}\cdot\vb{c})(\vb{b}\cdot\vb{d})-(\vb{a}\cdot\vb{d})(\vb{b}\cdot\vb{c}) $
        \item $\big(\vb{a}\times(\vb{b}\times\vb{c})\big)_{i}\ =\big((\vb{a}\cdot\vb{c})\vb{b}-(\vb{a}\cdot\vb{b})\vb{c}\big)_{i} $
        \item $\nabla f(r)=f'(r)\frac{\vb{r}}{r} $, $\grad r = \frac{\vb{r}}{r} $
        \item $\nabla f = \vb{e}_i\pdv{f}{x_j},\ \nabla\cdot\vb{F}=\pdv{F_j}{x_j},\ \nabla\times\vb{F}=\epsilon_{ijk}\vb{e}_i\pdv{F_k}{x_j} $
        \item $\|\vb{x}\|^2\|\vb{y}\|^2-|\vb{x}\cdot\vb{y}|^2\geq 0 $
        \item $\vb{F}\cdot\grad\neq-\grad\cdot\vb{F} $
        \item \includegraphics*[width=0.75\linewidth]{vec_identities.png}
        \item $\laplacian\vb{F}=\grad(\div\vb{F})-\curl(\curl\vb{F}) $
        \item $\curl(\grad\psi) = 0 $, $\div(\grad\vb{F}) = 0 $
        \item (Divergence Theorem) \begin{itemize}
            \item (Vector) $\iiint_{V}\div\vb{u}\dd V=\iint_{S}\vb{u}\cdot\dd {S} $
            \item (Scalar) $\iiint_{V}\grad\phi\dd V=\iint_{S}\phi\dd \vb{S} $
            \item (Generalized Stokes) $\iiint_{V}\curl\vb{A}\dd V = \iint_{S}\vb{\hat n}\times\vb{A}\dd S $
        \end{itemize}
        \item (Stokes Theorem) \begin{itemize}
            \item $\iint_{S}\curl\vb{u}\cdot\dd \vb{S} = \oint_{C}\vb{u}\cdot\dd\vb{r} $
            \item (Green's) $\iint_A\left(\pdv{u_y}{x}-\pdv{u_x}{y}\right)\dd x\dd y = \int_C(u_x\dd x+u_y\dd y) $
        \end{itemize}
        \item \includegraphics*[width=0.5\linewidth]{Curvilinear Coordinates.png}
        \item \includegraphics*[width=0.5\linewidth]{Curvilinear Coordinates2.png}
        \item \begin{itemize}
            \item $\vb{h}_j = \pdv{\vb{r}}{q_j} = \pdv{x_i}{q_j}\vb{\hat x}_i $
            \item $\vb{h}_j = h_j\vb{e}_j $, $h_j = \left|\pdv{\vb{r}}{q_j}\right| $
            \item $\vb{e}_j = \frac{1}{h_j}\pdv{\vb{r}}{q_j} $
            \item $J\equiv\pdv{(x,y,z)}{(q_1,q_2,q_3)}=|\vb{h}_1\cdot\vb{h}_2\times\vb{h}_3| $, $\dd V = \dd\vb{r}_1\times\dd\vb{r}_2\cdot\dd\vb{r}_3 = |J|\dd q_1\dd q_2\dd q_3 = h_1 h_2 h_3\dd q_1\dd q_2\dd q_3 $ (chain rule, inverse)
            \item (Orthogonality) $\vb{e}_i\cdot\vb{e}_j=\delta_{ij}, |\dd\vb{r}|^2=\sum_{i}h_i^2(\dd q_i)^2 $
            \item $\dd\vb{r}=\sum_i h_j\vb{e}_j\dd q_j $
            \item $\boxed{\grad=\sum_i\vb{e}_i\frac{1}{h_i}\pdv{q_i}} $
            \item (even permutation $(i,j,k)$) $\vb{e}_i=\vb{e}_j\times\vb{e}_k = h_j\grad q_j\times h_k\grad q_k $
            \item $\boxed{\div\vb{F}=\frac{1}{h_1 h_2 h_3}\sum_{\mathrm{even\ perms}}\pdv{h_j h_k F_i}{q_i}} $
            \item $\laplacian = \div\grad = \frac{1}{h_1 h_2 h_3}\sum_{\mathrm{even\ perms}}\pdv{q_i}\left(\frac{h_j h_k}{h_i}\pdv{q_i}\right) $
            \item $\boxed{\curl\vb{F}=\frac{1}{h_1 h_2 h_3}\begin{vmatrix}
                h_1\vb{e}_1 & h_2\vb{e}_2 & h_3\vb{e}_3\\
                \pdv{q_1} & \pdv{q_2} & \pdv{q_3}\\
                h_1F_1 & h_2F_2 & h_3F_3
            \end{vmatrix}} $
            \item $\laplacian\vb{F}=\laplacian(F_i\vb{e}_i) $  ($(\laplacian\vb{F})_i \neq \laplacian F_i $)
        \end{itemize}
    \end{enumerate}

    \section*{Green's functions}
    \begin{enumerate}
        \item Different $\delta_\epsilon(x)$s\begin{itemize}
            \item $\delta_\epsilon(x)=\begin{cases}
                0&x<-\epsilon\\
                \frac{1}{2\epsilon}&\quad -\epsilon\leq x\leq\epsilon\\
                0&\epsilon<x
            \end{cases}$
            \item $\delta_\epsilon(x)=\begin{cases}
                (x+\epsilon)/\epsilon^2 & -\epsilon<x<0\\
                (\epsilon-x)/\epsilon^2 & 0\leq x<\epsilon\\
                0 & \qotherwise
            \end{cases}$
            \item $\delta_\epsilon(x)=\frac{\epsilon}{\pi(x^2+\epsilon^2)}=\frac{1}{2\pi}\int_{\infty}^{-\infty}e^{ikx-\epsilon|k|\dd k}  \dd $
            \item 
        \end{itemize}
        \item $\delta(x)$ properties\begin{itemize}
                \item $\int_{-\alpha}^{\beta} \delta(x) \dd x = 1,\ \alpha>0,\ \beta>0$
                \item $\int_{-\infty}^{\infty} \delta(x-\xi)f(x) \dd x = f(\xi)$
                \item $H'(x)=\delta(x)$
                \item $\int_{-\infty}^{\infty} \delta'(x)f(x) \dd x = -f'(x)$
                \item $\delta(ax) = \frac{\delta(x)}{|a|}$
                \item For a nice function $\delta(f(x)) = \sum_{i}\delta(\eval{\dv{f}{x}}_{x=x_i}(x-x_i)) = \sum_i\frac{\delta(x-x_i)}{\left|\eval{\dv{f}{x}}_{x=x_i}\right|}$
            \end{itemize}
        \item Wronskian $W[y_1,y_2]=\det\begin{pmatrix}
            y_1 & y_2\\
            y_1' & y_2'
        \end{pmatrix}\neq 0$ is the condition for linearly independent solutions
        \item \textbf{Initial and boundary conditions} is written as $Ay(a)+By'(a)=E$, $E=0$ means BC is \textit{homogeneous}.
        \item Differentail operator $L=\dv[2]{x}+p(x)\dv{x}+q(x)$ (turn $\mathcal{L}$ into this \textcolor{red}{standard form} first, \textcolor{red}{coefficient of 2nd order term} is 1)
        \item Green's function is the solution to $LG(x,\textcolor{red}{\zeta})=\delta(x-\textcolor{red}{\zeta})$
        \item $y(x) = \int_{a}^{b} G(x,\textcolor{red}{\zeta})f(\zeta) \dd \zeta$
        \item Green's function properties\begin{itemize}
            \item $G(x,\zeta)$ of $L$ shares the same boundary conditions as $y(x)$, which is the solution to $Ly(x)=f(x)$, by construction
            \item $G(x,\zeta)$ is a continuous function in $x$ and $\zeta$
            \item $\lim_{\epsilon\rightarrow0}\left[G(x,\textcolor{red}{\zeta})\right]_{x=\zeta-\epsilon}^{x=\zeta+\epsilon}=0$
            \item $\lim_{\epsilon\rightarrow0}\left[\pdv{G}{x}\right]_{x=\zeta-\epsilon}^{x=\zeta+\epsilon} = 1$ (or whatever the \textcolor{red}{coefficient of the 2nd order term is})
        \end{itemize} (Start from $\boxed{1=\lim_{\epsilon\rightarrow0}\int_{\zeta-\epsilon}^{\zeta+\epsilon}\delta(\zeta-x)\dd x}$, $\mathcal{L}$ is 2nd order, $p,q$ are continuous, and assume $p(x)$ is continuous \mylabel{greens}{$\bigstar$})
        \item Writing $G(x,\zeta)=\begin{cases}
            \alpha_-(\zeta) y_1(x)+\beta_-(\zeta) y_2(x)\quad& \qfor a\leq x<\zeta\\
            \alpha_+(\zeta) y_1(x)+\beta_+(\zeta) y_2(x)\quad& \qfor \zeta\leq x\leq b,
        \end{cases}$
        it follows immediately that $\begin{pmatrix}
            y_1(\zeta) & y_2(\zeta)\\
            y_1'(\zeta) & y_2'(\zeta)
        \end{pmatrix}\begin{pmatrix}
            \alpha_+(\zeta) - \alpha_-(\zeta)\\
            \beta_+(\zeta) - \beta_-(\zeta)
        \end{pmatrix}=\begin{pmatrix}
            0\\1
        \end{pmatrix}$, if the solutions are independent,
        $\alpha_+(\zeta)-\alpha_-(\zeta)=-\frac{y_2(\zeta)}{W(\zeta)}$ and $\beta_+(\zeta) - \beta_-(\zeta) = \frac{y_1(\zeta)}{W(\zeta)}$
    \end{enumerate}


    \section*{Partial differential equations}
    Consider separable solutions
    
    For $x^2y''+axy'+y=0$, try $y=x^r$

    \section*{Matrices}
    \begin{enumerate}
        \item Metric $G_{ij}=\vb{u}_i\cdot\vb{u}_j$\begin{itemize}
            \item $\vb{v}\cdot\vb{w}=\vb{v}^\dagger G\vb{w}$
            \item $G^\dagger = G$ (Hermitian)
            \item $v^\dagger G v\geq 0$ (positive definite)
        \end{itemize}
        \item \begin{itemize}
            \item $\det M = \prod_{i=1}^{n}\lambda_i$ ($\det(AB)=\det A\det B$)
            \item $\tr M = \sum_{i=1}^{n}\lambda_i$
            \item $\tr(M^n) = \tr(\Lambda^n)$ ($\tr(AB)=\tr(BA)$)
        \end{itemize}
        \item Unitary matrix $A^\dagger = A^{-1}$
        \item Normal matrix $AA^\dagger = A^\dagger A$
        \item Hermitian matrices\begin{itemize}
            \item The eigenvalues of an Hermitian matrix are real
            \item The eigenvectors of an Hermitian matrix with distinct eigenvalues are orthogonal
            \item A Hermitian matrix has $n$ orthogonal linearly independent eigenvectors
            \item has $n$ orthonormal eigenvectors
            \item Anti-Hermitian and Unitary matrices have imaginary eigenvalues with unit modulus
        \end{itemize}
        \item Simplifying quadric surface $x^T A x+b^T x +c=0$\begin{itemize}
            \item $S=\frac{1}{2}(A+A^T)$, $y^T Sy+b^T y +c=0$ (symmetric thus diagonalizable)
            \item Diagonalize, $z^T \Lambda z+b^T z+c=0$
            \item Offset, $x'^T\Lambda x' = k$ (to cancel out second term)
        \end{itemize}
        \item Quadric surface names
        \begin{center}
            \includegraphics*[width=0.8\linewidth]{Quadric surfaces.png}
        \end{center}
        \item All eigenvalues of a \textcolor{red}{nilpotent} matrix are 0
        \item The Rayleight-Ritz variational principle. The first variation of $\lambda(x)=\frac{x^T S x}{x^T x}$ is 0 for all possible $\delta x$ when $Sx=\lambda(x)x$
    \end{enumerate}


    \section{Elementary analysis}
    \begin{enumerate}
        \item Limit, series, partial sum, absolute convergence$\implies$ (conditional) convergence
        \item Tests for convergence\begin{itemize}
            \item Comparison test (between two series)
            \item D'Alembert's ratio test
            \item Cauchy's test $\lim_{r\rightarrow\infty}u_r^{1/r} < 1$
        \end{itemize}
        \item Taylor $f(x_0+h)=\sum_{n=0}^{\infty}\frac{h^n}{n!}f^{(n)}(x_0)$
        \item $\dv{f}{z}\equiv f'(z)=\lim_{\delta z\rightarrow 0}\frac{f(z+\delta z)-f(z)}{\delta z}$ same by any route in the complex plane
        \item Cauchy-Riemann equations $f(z)=u(x,y)+iv(x,y)$, $\pdv{u}{x}=\pdv{v}{y}, \pdv{v}{x}=-\pdv{u}{y}$\begin{itemize}
            \item $u,v$ are harmonic functions, $\laplacian{u}=\laplacian{v}=0$
            \item $u,v$ are conjugate harmonic functions, $\grad{u}\cdot\grad{v}=0$
        \end{itemize}
        \item $C^1$ complex functions are analytic
        \item $f(z)~(z-z_0)^N$, zero of order $N$ at $z_0$, $f(z)~(z-z_0)^{-N}$, pole of order $N$
        \item Laurent series $f(z)=\sum_{-\infty}^{\infty}a_n(z-z_0)^n$, on annulus $\alpha<|z-z_0|<\beta$, infinite $n<0$ means essential singularity
        \item Radius of convergence: power series $f(z)=\sum_{r=0}^{\infty}a_r z^r$ converges for $z=z_1$, then it converges absolutely for $|z-z_0|<|z_1-z_0|$\begin{itemize}
            \item $\lim_{r\rightarrow\infty}\left|\frac{a_{r+1}}{a_r}\right| = \frac{1}{R}$
            \item $\lim_{r\rightarrow\infty}|a_r|^{1/r}=\frac{1}{R}$
        \end{itemize}
    \end{enumerate}
    

    \section*{Series solution of ODE}
    \begin{enumerate}
        \item Wronskian $W=\begin{vmatrix}
            y_1 & y_2\\
            y_1' & y_2'
        \end{vmatrix}$
        \item $W'+p(x)W=0$, $W(x)=C\exp\left(-\int^x p(\zeta)\dd\zeta\right)$, 
        
        $y_2(x)=y_1(x)\int^x\frac{W(\eta)}{y_1(\eta)^2}\dd\eta = y_1(x)\int^x\frac{C}{y_1(\eta)^2}\exp\left(-\int^\eta p(\zeta)\dd\zeta\right)$
        \item $y''+p(z)y'+q(z)=0$
        \item Ordinary point $p(z),\ q(z)$ analytic at $z=z_0$, regular singular point $(z-z_0)p(z),(z-z_0)^2q(z)$ analytic at $z=z_0$
        \item $z=z_0$ is ordinary point, two independent solutions like $y=\sum_{n=0}^{\infty}a_n(z-z_0)^n, \quad |z-z_0|<R$
        \item $z=z_0$ is regular singular point, $y_1=z^{\sigma_1}\sum_{n=0}^{\infty}a_n (z-z_0)^n,\quad a_0\neq 0, \sigma\in\mathbb{C}$
        \item Indicial equation $\sigma(\sigma-1)+p_0\sigma+q_0 = 0$, $p_0=\lim_{z\rightarrow z_0}((z-z_0)p(z))$, $q_0=\lim_{z\rightarrow z_0}((z-z_0)^2q(z))$
        \item If $\sigma_1-\sigma_2\in \mathbb{R}$, $\mathrm{Re}(\sigma_1)\geq\mathrm{Re}(\sigma_2)$ another solution is \textcolor{red}{$y_2 = ky_1\log z + z^{\sigma_2}\sum_{0}^{\infty}b_n z^n$}
        \item Variation of parameters 
    \end{enumerate}

    \section*{Sturm-Liouville}
    \begin{enumerate}
        \item Norm of $u(x)$ is $||u||^2 = \langle u|u\rangle = \int_\alpha^\beta|u(x)|^2\dd x$ is real and $\geq0$.
        \item If $\langle u|\mathcal{L}v\rangle = \langle \mathcal{L} u|v\rangle = \langle v|\mathcal{L}u\rangle^*$ if boundary terms are 0, called \textbf{self-adjoint}.
        \item Sturm-Liouville operator defined on $\alpha\leq x\leq\beta$ is $\mathcal{L}=-\dv{x}\left(\rho(x)\dv{x}\right)+\sigma(x)$, $\sigma,\rho$ are real, $\forall\alpha<x<\beta(\rho>0)$\begin{itemize}
            \item $\langle u|\mathcal{L}v\rangle = \langle v|\mathcal{L} u\rangle^* + [\rho(v{u^{*}}'-u^*v')]^\beta_\alpha$ means formally self-adjoint (differ by a constant)
            \item If $[\rho(v{u^{*}}'-u^*v')]^\beta_\alpha = 0$, $\mathcal{L}$ is self-adjoint
            \item Might not work if $\mathcal{L}$ not defined on $x\in[\alpha,\beta]$ e.g.$[-1,1]$ for Legendre's equation $(1-x^2)y''-2xy'+l(l+1)y=0$, solutions $x$ and $\frac{1}{2}\ln(\frac{1+x}{1-x})$ not orthogonal
        \end{itemize}
        \item Inner product with a weight function $\langle u|v\rangle_w = \int_\alpha^\beta w(x)u^*(x)v(x)\dd x$
        \item $\mathcal{L}=w\tilde{\mathcal{L}}$ for a second order operator $\tilde{\mathcal{L}}=-\dv{x}\left[a(x)\dv{x}\right]-b(x)\dv{x}-c(x)$\begin{itemize}
            \item $w(x)$ is real and positive
            \item $w\tilde{\mathcal{L}}=-\dv{x}\left(aw\dv{x}\right)+(aw'-bw)\dv{x}-wc$
            \item Let $aw'=bw$, $w(x)=Ce^{\int^x\frac{b(\zeta)}{a(\zeta)}\dd \zeta}$
            \item Note that $w(x),a(x),b(x),c(x)$ are real (by definition)
            \item $\langle u|\tilde{\mathcal{L}}v\rangle_w = \langle \tilde{\mathcal{L}} v| u\rangle_w + [wa(v{u^{*}}'-u^*v')]^\beta_\alpha$
            \item i.e. $\boxed{(\lambda_u-\lambda_v)\int_a^b u^*vw\dd x=\bigg[\rho(x)({u^*}'v'-u^*v')\bigg]_a^b}$
            \item $\tilde{\mathcal{L}} y=\lambda y\implies \mathcal{L}y=\lambda wy$
        \end{itemize}
        \item If $\{y_n\}$ is a complete set of \textbf{orthonormal} eigenfunctions,
        
        the completeness relation is $\boxed{\sum_{n=1}^{\infty}y_{n}(x)y_{n}^{*}(x^{\color{red}{'}})={\frac{1}{w(x^{\prime})}}\delta(x-x^{\color{red}{'}})}$
        \begin{itemize}
            \item $f(x)=\sum_{n=1}^{\infty}a_n y_n(x)$
            \item $\langle y_n|y_m\rangle_w = \delta_{nm}$
            \item $a_n = \langle y_n|f\rangle_w$
            \item $f(x)=\sum_{n=1}^{\infty}\langle y_n|f\rangle_w y_n(x) = \int_{\alpha}^{\beta}f(x^{\prime})\left[w(x^{\prime})\sum_{n=1}^{\infty}y_{n}(x)y_{n}^{*}(x^{\prime})\right]\,d x^{\prime}$
            \item Recall that $\mathcal{L}G(x,x')=\delta(x-x')$, $y(x)=\int_\alpha^\beta G(x,x')f(x')\dd x'$
            \item $\boxed{G(x,x')=\sum_{n=1}^{\infty}\frac{1}{\lambda_n}y_n(x)y_n^*(x')}$ such that $\mathcal{L}G(x,x')=w(x)\sum_{n=1}^{\infty}y_n(x)y^*_n(x')=\frac{w(x)}{w(x')}\delta(x-x')=\delta(x-x')$ ($\mathcal{L}=\mathcal{L}_x$ acts on $y(x)$ only instead of $y^*(x')$)
            \item $G(x,x')=G^*(x,x')$, if $\mathcal{L}$ has zero eigenvalue $G(x,x')$ will not exist
        \end{itemize}
        \item Bessel's inequality $\|f\|^2_w\geq\sum_{n=1}^{N}|a_n|^2$\begin{itemize}
            \item $f(x)\approx\sum_{n=1}^{N}a_ny_n$, let $a_n=u+iv$
            \item Error 
            \begin{align*}
                E &= \left|f(x)-\sum_{n=1}^{N}a_ny_n\right|^2_w\\
                  &= \|f\|^2_w-\sum_{n=1}^{N}[a^*_n\langle y_n|f\rangle_w+a_n\langle f|y_n\rangle_w] + \sum_{n=1}^{N}\sum_{n=1}^{N}a^*_n a_m\langle y_n|y_m\rangle_w \\
                  &= \|f\|^2_w-\sum_{n=1}^{N}[(u-iv)(\langle y_n|f\rangle_w)^* + (u+iv)\langle f|y_n\rangle_w] + \sum_{n=1}^{N}|a_n|^2 \\
                  &= \|f\|^2_w + \sum_{n=1}^{N}(u^2+v^2) - \sum_{n=1}^{N}[2u\Re{\langle y_n|f\rangle_w}+2v\Im{\langle y_n|f\rangle_w}]
            \end{align*}
            \item If $\pdv{E}{u} = 0 = \pdv{E}{v}$, $a_n=\langle y_n|f\rangle_w$, $E = \|f\|^2_w - \sum_{n=1}^{N}|a_n|^2 \geq 0$, becomes equality at $N\rightarrow\infty$
        \end{itemize}
    \end{enumerate}

    \section*{Calculus of variation}
    \begin{enumerate}
        \item The functional $G[y] : \{y_k\}\rightarrow\mathbb{R}=\int_\alpha^\beta f(y,y';x)\dd x$
        \item $\delta G=\int_\alpha^\beta\delta y(x)\frac{\delta G}{\delta y(x)}\dd x+\ldots = \left[\delta y\pdv{f}{y'}\right]_{\alpha}^{\beta}+\int_{\alpha}^{\beta}\delta y\left[\pdv{f}{y}-\dv{x}\left(\pdv{f}{y'}\right)\right]\dd x + \ldots$
        \item Euler Lagrange equation $\frac{\delta G}{\delta y(x)} = 0 = \pdv{f}{y}-\dv{x}\left(\pdv{f}{y'}\right)$, $\pdv{f}{y} = \dv{x}\left(\pdv{f}{y'}\right)$
        \item \textbf{First integral} if integrand $f(y,y',x)=f(y,y')$ does not depend on $x$,\begin{align*}
            \dv{f}{x}&=\pdv{f}{x}+y'\pdv{f}{y}+y''\pdv{f}{y'}\\
                     &=\pdv{f}{x}+\dv{x}\left(y'\pdv{f}{y'}\right)\\
            \dv{x}\left(f-y'\pdv{f}{y'}\right)&=\pdv{f}{x}=0\\
        \end{align*}\begin{center}
            $\boxed{y'\pdv{f}{y'}-f=c}$
        \end{center}
        \item Sturm-Liouville\begin{itemize}
            \item $F[y]=\langle y|\mathcal{L}y\rangle = \int_\alpha^\beta y^*\mathcal{L}y \dd x = \int_\alpha^\beta \left[\rho|y'|^2+\sigma |y|^2\right] \dd x$
            \item $G[y]= \langle y|y\rangle_w = \int_\alpha^\beta w|y|^2 \dd x$
            \item $\frac{\delta F}{\delta y}=2\mathcal{L}y$, $\frac{\delta G}{\delta y}=2wy$
            \item $\Lambda[y] = \frac{\langle y|\mathcal{L}y\rangle}{\langle y|y\rangle_w} =\frac{F[y]}{G[y]}$
            \item $\frac{\delta \Lambda}{\delta y}=\frac{1}{G}\left[\frac{\delta F}{\delta y}-\Lambda \frac{\delta G}{\delta y}\right]=\frac{2}{G}\left[\mathcal{L}y-\Lambda wy\right]$
            \item At extremum, $\mathcal{L}y=\lambda wy=\Lambda[y] wy$, $\Lambda[y]$ is extremized by eigenfunctions of $\tilde{\mathcal{L}}=w^{-1}\mathcal{L}$ and the eigenvalues $\lambda$ are its extremal values
        \end{itemize}
        \item \begin{align*}
            \dv{L}{t} &=\pdv{L}{t}+\sum_{i=1}{N}\dot q_i\pdv{L}{q_i} + \ddot q_i\pdv{L}{\dot q_i}\\
                      &= \pdv{L}{t}+\sum_{i=1}{N}\dot q_i\dv{t}\pdv{L}{\dot q_i} + \ddot q_i\pdv{L}{\dot q_i}\\
                      &= \pdv{L}{t}+\dv{t}\sum_{i=1}{N}\dot q_i\pdv{L}{\dot q_i}\\
            \dv{t}\left[L+\sum_{i=1}^N\dot q_i\pdv{L}{\dot q_i}\right] &= \pdv{L}{t} = \dv{H}{t}
        \end{align*}
        \item Time invariance of $L$ is energy ($H$) conservation, $q$ invariance of $L$ is $\pdv{L}{\dot q}$ conservation (generalized momentum)
        \item $\phi(x,y,\lambda)=f(x,y)-\lambda p(x,y)$, $\Phi_\lambda[y]=G[y]-\lambda P[y]$
    \end{enumerate}

    \section*{Laplace Poisson}
    \begin{enumerate}
        \item Find constants with \textbf{boundary conditions}, implicit conditions for \textbf{physical} solutions, \textbf{continuity of derivative} and \textbf{continuity of solution}
        \item Poisson's equation $\laplacian\Phi=\rho(\vb{x})$, $\rho=0$ is Laplace's equation\begin{itemize}
            \item Diffusion $\kappa\laplacian u=\pdv{u}{t}-S(\vb{x})$, in steady state $\pdv{u}{t}=0$. $\laplacian{u}=-\frac{S(\vb{x})}{\kappa}$, \textcolor{red}{flux $\vb{F}=-\kappa\grad u$}
            \item $\div\vb{E}=\rho_q(\vb{x})/\epsilon_0$, $\curl\vb{E}=0\implies\vb{E}=-\grad\Phi$, $\laplacian{\Phi}=-\rho_q(\vb{x})\epsilon_0$
            \item $\laplacian{\Phi}=4\pi G\rho_m(\vb{x})$
            \item 3D Schr\"odinger
            \item Ideal fluid (irrotational, incompressible) $\curl u=0\implies u=\grad\Phi$, continuity (incompressible) $\rho\div u=-\frac{\rho}{t}=S(\vb{x})$,$\laplacian{\Phi}=0$
        \end{itemize}
        \item Cylindrical $\Psi = \Psi(r,\phi)$, $x=r\cos\phi,y=r\sin\phi$, $\laplacian{\Psi}=\frac{1}{r}\pdv{r}\left(r\pdv{\Psi}{r}\right)+\frac{1}{r^2}\pdv[2]{\Psi}{\phi}=0$\[
            \boxed{\Psi = A_0+B_0\textcolor{red}{\phi}+C_0\textcolor{red}{\ln r}+D_0\textcolor{red}{\phi\ln} r+\sum_{n=1}^{\infty}(A_n r^{\textcolor{red}{n}}+C_n r^{\textcolor{red}{-n}})\cos \textcolor{red}{n}\phi+\sum_{n=1}^{\infty}(B_n r^n+D_n r^{-n})\textcolor{red}{\sin n\phi}}
        \]
        \item Axisymmetric Spherical $\Psi=\Psi(r,\theta)$, $x=r\sin\theta\cos\phi, y=r\sin\theta\sin\phi, z=r\cos\theta$, $\laplacian\Psi=\frac{1}{r^2}\pdv{r}\left(r^2\pdv{\Psi}{r}\right)+\frac{1}{r^2\sin\theta}\pdv{\theta}\left(\sin\theta\pdv{\Psi}{\theta}\right)=0$\[
            \boxed{\Psi(r,\theta)=\sum_{l=0}^{\infty}(A_l r^{\textcolor{red}{l}}+B_l r^{\textcolor{red}{-l-1}})P_l(\textcolor{red}{\cos\theta})}
        \]
        \item \begin{tabular}{|Sc|c|c|}
                $P_1(x)$ & $1$\\
                $P_2(x)$ & $x$\\
                $P_3(x)$ & $\frac{1}{2}(\textcolor{red}{3}x^2-\textcolor{red}{1})$\\
                $P_4(x)$ & $\frac{1}{2}(\textcolor{red}{5}x^3-\textcolor{red}{3}x)$\\
            \end{tabular}
        \item The solution to Poisson's equation is unique with Dirichlet ($\Phi(\vb{r})=f(\vb{r})$) or Neumann ($\pdv{\Phi}{n}=\vb{n}\cdot\grad{\Phi}=f(\vb{r})$) boundary conditions on a surface $S$.
        
        Prove using difference of two solutions and $\div(\Psi\grad\Psi)=\grad\Psi\cdot\grad\Psi+\Psi\div(\grad\Psi)$
        \item Green's function \[
            \begin{cases}
                \laplacian{G(\vb{r},\vb{r}')}_{\vb{r}}=\delta^{(3)}(\vb{r}-\vb{r}'),&\quad \vb{r}\ in\ V\\
                (\mathrm{Dirichlet})\ G=0,&\quad \vb{r}\ on\ S\\
                (\mathrm{Neumann})\ \pdv{G}{n}=\frac{1}{A},&\quad \vb{r}\ on\ S
            \end{cases}
        \]\where{$A=\oint_S\dd S$}
        If $V$ is all of space, $G$ is the \textbf{fundamental solution}.
        \item The fundamental solution in 3D: $\laplacian{G}=\delta^{(3)}(\vb{r}-\vb{r}')$, $G\rightarrow 0 \qas |\vb{r}|\rightarrow\infty$\begin{itemize}
            \item Spherically symmetric, $G=G(\vb{r})$
            \item $\laplacian{G}=(r^2 G')'/r^2=0,\ G=\frac{C}{r}+A$, $A=0$
            \item $S$ is a sphere of radius $\epsilon$; $\eval{\pdv{G}{r}}_{r=\epsilon}=-\frac{C}{\epsilon^2}$
            \item $\int_{r<\epsilon}\laplacian{G}\dd V=\oint_{r=\epsilon}\grad G\cdot\vb{n}\dd S=-\frac{C}{\epsilon^2}\oint_{r=\epsilon}\dd S=-4\pi C$
            \item $\laplacian{G}=\delta^{(3)}(\vb{r}-\vb{r}')=-4\pi C\delta^{(3)}(\vb{r}-\vb{r}')$, $C=-\frac{1}{4\pi}$
            \item $\boxed{G(\vb{r},\vb{r}')=-\frac{1}{4\pi|\vb{r}-\vb{r}'|}}$
        \end{itemize}
        \item The fundamental solution in 2D: $\laplacian{G}=\delta^{(2)}(\vb{r}-\vb{r}')$, $|\grad G|\rightarrow0\ \qas\ |\vb{r}|\rightarrow \infty$ (or $G$ vanishes in a finite radius)\begin{itemize}
            \item Circularly symmetric, $G=G(\vb{r})$
            \item $(rG')'/r=0\implies G=C\ln r+A$, $r\neq 0$
            \item $S$ is a circle with radius $\epsilon$; $\eval{\pdv{G}{r}}_{r=\epsilon} = \frac{C}{\epsilon}$
            \item $\int_{r<\epsilon}\laplacian{G}\dd A=\oint_{r\epsilon}\grad G\cdot\vb{n}\dd l=\frac{C}{\epsilon}\oint_{r=\epsilon}\dd l=2\pi C$
            \item $\laplacian{G}=\delta^{(2)}(\vb{r}-\vb{r}')=2\pi C\delta^{(2)}(\vb{r}-\vb{r}')$, $C=\frac{1}{2\pi}$
            \item $\boxed{G(\vb{r},\vb{r}')=\frac{1}{2\pi}\ln|\vb{r}-\vb{r}'|+C}$
        \end{itemize}
        \item Method of images: \textbf{Dirichlet} boundary condition --- charge of \textbf{opposite} sign, to cancel out at the boundary; \textbf{Neumann} --- charge of \textbf{same} sign, to make $\pdv{G}{n}=0$ at the boundary
        \item For sphere and circle, it is equivalent to a inverse point with a particular strength
        \item $\vb{F}=\Phi\grad\Psi-\Psi\grad\Phi$, $\div(\Phi\grad\Psi)=\grad\Phi\cdot\grad\Psi+\Phi\laplacian{\Psi}$ leads to Green's theorem\[
            \boxed{\int_V(\Phi\laplacian{\Psi}-\Psi\laplacian{\Phi})\dd V=\oint_S(\Phi\grad\Psi-\Psi\grad\Phi)\cdot\dd S=\oint_S\left(\Phi\pdv{\Psi}{n}-\Psi\pdv{\Phi}{n}\right)}
        \] (replace $V$ and $S$ to $A$ and $l$ in 2D)\begin{itemize}
            \item (Dirichlet) For $\begin{cases}
                \laplacian{\Phi}&=\rho(\vb{r}),\quad \mathrm{in}\ V\\
                \Phi(\vb{r})&=f(\vb{r}),\quad \mathrm{on}\ S\\
            \end{cases}$
            \item Let \textcolor{red}{$\Psi=G$}, \begin{align*}
                \int_V (\Phi\laplacian{G}-G\laplacian{\Phi})\dd V &= \oint_S(\Phi\grad G-G\grad\Phi)\cdot\vb{n}\dd S\\
                \int_V\Phi\delta^{(3)}(\vb{r}-\vb{r}')\dd V &= \int_V G\rho\dd V+\oint_S f\pdv{G}{n}\dd S\\
                \textcolor{red}{\Phi(\vb{r'})} &= \int_V \rho(\vb{r})G(\vb{r},\vb{r}')\dd V+\oint_S f(\vb{r})\pdv{G}{n}\dd S\\
                \textcolor{red}{\Phi(\vb{r'})} &= \int_{\real^3} \rho(\vb{r})G(\vb{r},\vb{r}')\dd V = \int_{\real^3}\frac{\rho_q(\vb{r})}{4\pi\epsilon_0|\vb{r}-\vb{r}'|}\dd V
            \end{align*}
            The last line is for all space (sphere of radius $\infty$)

            Turn it into Laplace by setting $\forall \vb{r}\in V(\rho(\vb{r})=0)$
            \item (Neumann) For $\begin{cases}
                \laplacian{\Phi}&=\rho(\vb{r}),\quad \mathrm{in}\ V\\
                \pdv{\Phi}{n}&=f(\vb{r}),\quad \mathrm{on}\ S\\
            \end{cases}$
            \item Let $G=\Psi$. $\eval{\pdv{G(\vb{r},\vb{r}')}{n}}_{\vb{r}\in S}=\frac{1}{A}$, \begin{align*}
                \Phi(\vb{r}') &=\int_V\rho(\vb{r})G(\vb{r},\vb{r}')\dd V+\frac{1}{A}\oint_S\Phi(\vb{r})\dd S-\oint_S f(\vb{r})G(\vb{r},\vb{r}')\dd S\\
                \Phi(\vb{r}') &= \int_V\rho(\vb{r})G(\vb{r},\vb{r}')\dd V-\oint f(\vb{r})G(\vb{r},\vb{r}')\dd S
            \end{align*}
            \where{the second line follows from finite surface integral of $\Phi(\vb{r})$ and infinite $A$ of $V$ being all space}
        \end{itemize}
    \end{enumerate}

\section*{Cartesian tensors}
    \begin{enumerate}
        \item A vector $\vb{v}$ is a set of numbers $v_i$ defined wrt. a set of orthonormal basis vectors $\vb{e}_i$ by $\boxed{v_i'=L_{ij}v_j}$, where $L_{ij}=\vb{e}_i'\cdot\vb{e}_j$ (\fbox{$L_{ij}$ is \textit{orthogonal}, $L^T L=I$})
        \item $\grad=\vb{e}_i\partial_i$, $\partial_i\equiv\pdv{x_i}$, $\partial{x_i'}=L_{ij}\partial{x_j}$ (by chain rule) ($\grad$ is therefore a vector, but only in Cartesian coordinates, where covectors same as vectors)
        \item a Cartesian axial vector (pseudo-vector) $a$ is a set of coefficients $a_i$ defined wrt. a set of orthonormal basis vectors $\vb{e}_i$, s.t. $a_i'$ wrt. another orthonormal basis $\vb{e}_i'$ are given by $a'_i=\det(L)L_{ij}a_j$ (same as vector under proper transformation $\det(L)=1$. Not reversed after improper rotations $\det(L)=-1$)
        \item A Cartesian tensor T of order $n$ has $n$ indices $T_{i_n\ldots i_n}$, defined wrt. a set of orthonormal basis vectors $\vb{e}_i$ that transforms like $\vb{e}_i'=L_{ij}\vb{e}_j$. $\boxed{T'_{i_1\ldots i_n}=L_{i_1j_1}\ldots L_{i_n j_n}T_{i_1\ldots i_n}}$
        \item Likewise, a Cartesian pseudo-tensor $E$ of order $n$ is $\boxed{E'_{i_1\ldots i_n}=\det(L)L_{i_1j_1}\ldots L_{i_n j_n}E_{i_1\ldots i_n}}$
        \item $\delta_{ij}$ is a tensor, $\epsilon_{ijk}$ is a pseudo-tensor
        \item Linear combination of order $n$ tensors are order $n$ tensors (closed under addition)
        \item Tensor $n$ $\otimes$ tensor $m\rightarrow$ tensor $n+m$; Tensor $n$ $\otimes$ pseudo-tensor $m\rightarrow$ pseudo-tensor $n+m$
        \item Contraction reduces order $n$ to $n-2$
        \item \textbf{Symmetric} - same after swapping two indices; \textbf{Antisymmetric} - $\times(-1)$ after swapping. Symmetry is invariant of coordinate system. For symmetric $T_{ijk}$, antisymmetric $E_{pqr}$, \newline
        $T_{ijk}E_{ijr}=0$
        \item Any 2nd order tensor expressed as symmetric and antisymmetric tensors $T_{ij}=S_{ij}+A_{ij}$, $S_{ij}=(T_{ij}+T_{ji})/2$, $A_{ij}=(T_{ij}-T_{ji})/2$
        \item Pseudo-vectors are equivalent to 2nd order antisymmetric tensors $\omega_k=\frac{1}{2}\epsilon_{ijk}A_{jk}$, \[A_{ij}=\epsilon_{ijk}\omega_k = \begin{vmatrix}
            0 & \omega_3 & -\omega_2\\
            -\omega_3 & 0 & \omega_1\\
            \omega_2 & -\omega_1 & 0
        \end{vmatrix}\]
        \item Symmetric tensors are sums of a constant and a traceless symmetric tensor $\tilde{S}=S-\frac{\Tr(S)}{3}\mathbb{I}$, $\Tr(\tilde{S})=0$
        \item Isotropic tensors have same components in all frames $\boxed{T'_{ijk\ldots}=T_{ijk\ldots}}$. The most general isotropic tensors:\begin{itemize}
            \item 0th order, scalar (all)
            \item 1th order, only $\vb{0}$ (for both vector \& pseudovector)
            \item 2nd order, $\lambda\delta_{ij}$, $\lambda$ is a scalar
            \item 3rd order, $\lambda\epsilon_{ijk}$
            \item 4th order, $\lambda\delta_{ij}\delta_{kl}+\mu\delta_{ik}\delta_{jl}+\nu\delta_{il}\delta_{jk}$
        \end{itemize}
        \item isotropic $\neq$ homogeneous
    \end{enumerate}

\section*{Contour integration}
    \begin{enumerate}
        \item Complex derivative $\dv{f}{z}=\lim_{\delta z\rightarrow0}\frac{f(z+\delta z)-f(z)}{\delta z}$ is the same for all $\delta z$, with $f(z)=u(x,y)+iv(x,y)$, the \textit{Cauchy-Riemann equations} are \[
                \pdv{u}{x}=\pdv{v}{y}\quad\pdv{v}{x}=-\pdv{u}{y}
            \]
            (Or simply $\pdv{f}{z^*}=0=\pdv{(u+iv)}{(x-iy)}$), and $u,v$ are harmonic, $\laplacian{u}=\laplacian{v}=0$, and conjugate harmonic $\grad u\cdot\grad v=0$
        \item Cauchy's theorem $\boxed{\oint_C f(z)\dd z=0}$ on a simply connected region $R$ without singularity/$f(z)$ being analytic ($C^1$); path independent from $a$ to $b$, $\int_{C_1}f(z)\dd z=\int_{C_2}f(z)\dd z$, if no singularity between $C_1,C_2$
        \item Laurent series $\boxed{f(z)=\sum_{n=-\infty}^{\infty}a_n(z-z_0)^n}$, on \textbf{annulus} $\alpha<|z-z_0|<\beta$
        \item The \textbf{residue} of a pole is $a_{-1}$ (coefficient in Laurent series)  For a simple pole, $\mathrm{res}_{z=z_0}f(z)=\lim_{z\rightarrow z_0}\left[(z-z_0)f(z)\right]$; for an $N$-pole, $\boxed{\mathrm{res}_{z=z_0}=\lim_{z\rightarrow z_0}\left[\frac{1}{(N-1)!}\dv{^{(N-1)}}{z^{N-1}}[(z-z_0)^Nf(z)]\right]}$; or just use Laurent series (for essential singularity)
        \item Residue theorem $\boxed{\oint_C f(z)\dd z=2\pi i\sum_{k=1}^n\mathrm{res}f(z_k)}$, $C$ is \textbf{anticlockwise} and encloses all $z_k$ (hint: Laurent series' negative powers around singularity $z_k$, shrink $C$ to a circle around it/or $\oint_Cf(z)\dd z$ subtract (or add clockwise) integrals of these circles $=0$)
        \item For \emph{analytic} $f(z)$ on $R$, Cauchy's formula $\boxed{f(z_0)=\frac{1}{2\pi i}\oint_C\frac{f(z)}{z-z_0}\dd z}$, $f^{(n)}(z_0)=\frac{n!}{2\pi i}\oint_C\frac{f(z)}{(z-z_0)^{n+1}}\dd z$, where simple closed anticlockwise $C$. So \emph{analytic complex function is infinitely differentiable}.
        \item Branch cut for multivalued functions \begin{itemize}
                \item $\ln(z)=\ln r+i(2n\pi+\theta)$
                \item $z^{1/a}=re^{i(2n\pi+\theta)/a},\ a>1$
            \end{itemize}
        \item Jordan's lemma $\lim_{R\rightarrow\infty}\int_{C_R}g(z)e^{i\lambda z}\dd z=0$ if \begin{itemize}
                \item $g(z)\rightarrow0$
                \item $C_R$ is upper semicircle, $\lambda>0$, or
                \item $C_R$ is lower semicircle, $\lambda<0$
            \end{itemize}
        \item Very important inequality in complex analysis \textcolor{red}{$0<\theta<\frac{\pi}{2}$, $\sin x>\frac{2x}{\pi}$}
            \begin{center}
                \includegraphics*[width=0.4\linewidth]{complex_analysis_inequality.png}
            \end{center}
        \item Gaussian integration lemma $\forall a\in\mathbb{C},\ \int_{-\infty}^{\infty}e^{-(u+a)^2}\dd u=\sqrt{\pi}$
    \end{enumerate}
\section*{Small Oscillations}
    \begin{enumerate}
        \item $\mathcal{L}=T-V=\frac{1}{2}T_{ij}q_iq_j-\frac{1}{2}V_{ij}q_iq_j$
        \item $T_{ij}\ddot{q}_j+V_{ij}q_j=0$
        \item $(-\omega^2\vb{T}+\vb{V})\vb{Q}=\vb{0}$, $\vb{q}=\vb{Q}\sin(\omega(t-t_0))$ or $Q_i=q_i(t-t_0)$ if $\omega=0$
        \item $\vb{Q}$ is \textbf{generalized engenvector}. $\alpha^{(n)}(t) = Q^{(n)}_iT_{ij}q_j(t)=A^{(n)}\sin\omega_n(t-t_0^{(n)})$ is a \textbf{normal coordinate}. (A variable substitution to get a simple $\textstyle\sin(t-t_0^{(n)})$)
        \item \textbf{Orthonormality} $(\vb{Q}^{(m)})^T\vb{T}\vb{Q}^{(n)}=\delta_{mn}$
    \end{enumerate}
\section*{Group theory}
    \begin{enumerate}
        \item A group $(G,*)$ is a set $G$ and a binary operation $*$ satisfying\begin{itemize}
                \item Identity axiom, $\forall g\in G\exists e,\ e*g=g*e=g$
                \item Inverses axiom, $\forall g\in G\exists h\in G,\ h*g=g*h=e$
                \item Associativity axiom, $\forall g,h,k\in G,\ (g*h)*k=g*(h*k)$
                \item Closedness/Closure, $g_i* g_j\in G$
            \end{itemize}
        \item Examples\begin{itemize}
                \item nonzero complex numbers under multiplication $(\mathbb{C},\times)$
                \item $2\times2$ invertible matrices/general linear group of degree 2 over real numbers under multiplication $(GL_2(\mathbb{R}),*)$
                \item $2\times2$ real matrices under addition $M_2(\mathbb{R}, +)$
                \item Integers under addition $(\mathbb{Z},+)$
                \item \textbf{Symmetric groups}/\textbf{general permutation groups} $S_n$/$\Sigma_N$, $|S_n|=n!$, defined by $(S,*)$, where $S$ is the set of all permutations of $\{1,2,\ldots,n\}$, $*$ is a bijection from $\{1,2,\ldots,n\}$ to itself.
                \item $U(n)$, unitary $n\times n$ matrices
                \item $GL(n,\mathbb{C})$, $n\times n$ invertible matrices
                \item $S_n$ symmetric group/permutation group, $|S_n|=n!$
                \item $D_n$ dihedral group, $|D_n|=2n$, $n$ rotations by $\frac{360^\circ}{n}$, $n$ reflections
                \item Small abelian and non-abelian groups (\href{https://en.wikipedia.org/wiki/List_of_small_groups}{link})
            \end{itemize}
        \item \textbf{Order} of a group $G$ - number of elements in it
        \item $g^q=e$, $q$ is the \textbf{order} of group element $g$ (if does not exist, infinite order)
        \item A \textbf{cyclic} group satisfies $G=\{g^n:n\in\mathbb{Z}\}$
        \item $g_1,g_2$ are \textbf{generators} if $G=\{\prod_{n}g_n:n\in\mathbb{Z},g_n\in\{g_1,g_2\}\}$
        \item A group is \textbf{abelian} if every two elements commute. Cyclic groups are abelian.
        \item \includegraphics*[width=0.7\linewidth]{group_ring_field.jpg}
        \item $D_n$, $n$-fold dihedral groups, order $2n$. reflections about diagonal
        \item Group table: For $g_1g_2$, apply column $g_2$ then row $g_1$. Each row/column is a \textbf{complete rearrangement/derangements} of another.
        \item A subgroup is a subset that's also a group
        \item Let $(G,*)$, $(H,\times)$ be groups\begin{itemize}
            \item A \textbf{group homomorphism} $f:G\rightarrow H$ is a function such that $\forall x,y\in G,\ f(x*y)=f(x)\times f(y)$ (preserve group operation)
            \item A \textbf{group isomorphism} is a bijective group homomorphism
            \item (Example of homomorphism but not isomorphism: a subgroup, chosen carefully)
        \end{itemize}
        \item If $(H,\times)$ is an isomorphism of $(G,*)$, $H$ are $n\times n$ invertible matrices, $\times$ is matrix multiplication, group $H$ is a \textbf{faithfaul representation} of $G$
        \item Two elements are \textbf{conjugate}, $g_1\sim g_2$ iff $\exists g,\ g_2=gg_1g^{-1}$
        \item \textbf{Conjugacy classes} of a group are disjoint classes of elements, where the elements in each of them are mutually conjugate. The conjugacy class of $g$ is $\mathrm{Cl}(g)=\{hgh^{-1}|h\in G\}$
        \item Conjugacy classes for $D_n$ ($n$th order dihedral groups) are \begin{itemize}
                \item $\{e\}$
                \item $\{R^2,\ldots,R^n\}$
                \item $\{m_1,\ldots,m_n\}$
            \end{itemize}
        \item A \textbf{normal subgroup} $H$ is a subgroup that consists entirely of conjugacy classes of $G$. $G$ is a subgroup. A \textbf{proper normal subgroup} is such group with $H\neq G$. $I=e\in G$.
        \item A group $G$, a subgroup $H=\{I,h_1,h_2\ldots\}$ of $G$, $g\in G$, a \textbf{left coset} of $H$ in $G$ is $gH=\{g,gh_1,gh_2,\ldots\}$, a \textbf{right coset} of $H$ in $G$ is $Hg=\{g,h_1 g,h_2 g,\ldots\}$.
        \item Properties\begin{itemize}
                \item $(g_1g_2)^{-1}=g_2^{-1}g_1^{-1}$
                \item $g_1\sim g_2\implies g_2\sim g_1$
                \item $g_1\sim g_2, g_2\sim g_3\implies g_1\sim g_3$
                \item The identity $e$ of any group is a conjugacy class by itself
                \item Each element of an abelian group is in a class by itself
                \item $g$ and $g^{-1}$ may or may not be in the same conjugacy class
                \item The left and right coset of any subgroup $H$ of $G$ are identical if $G$ is abelian
                \item The left and right coset of any \emph{normal} subgroup $H$ are always identical ($gH=Hg$)\newline
                        $gh\in gH$, $ghg^{-1}=h_1\in H$ because $H$ is normal, $gh=h_1 g\in Hg$. vice versa.
                \item Order is the same in cosets $|gH|=|Hg|=|H|$ \newline
                        elements of cosets are still distinct
                \item Two cosets are either disjoint or identical \newline
                        For $Hg_1, Hg_2$, if $h_1g_1=h_2g_2$, $Hg_1=Hh_1^{-1}h_2g_2=Hg_2$ because $h_1^{-1}h_2\in H$
                \item Two cosets $Hg_1$ and $Hg_2$ are identical iff $g_1g^{-1}_2\in H$
                \item Every element of $G$ is in some coset
                \item The subgroup $H$ and its left cosets partition $G$\newline
                        Because $I\in H$, $\forall g\in G$, $g\in Hg$
                \item (\textbf{Lagrange's theorem}) \fbox{If $H$ is a subgroup of $G$, $|G|=n|H|,\ n\in\mathbb{Z}^+$}
                    \where{$n=|G:H|$ is the index of $H$ in $G$, \newline the number of distinct left/right cosets of $H$ in $G$}
                \item The order of every $g\in G$ divides $|G|$\newline
                        Each element generates a cyclic subgroup of the same order.\\
                        If $|G|$ is prime, $G=C_n$ the cyclic group.
            \end{itemize}
        \item All order 4 groups are isomorphic to the cyclic group $C_4$ or the \href{https://mathworld.wolfram.com/Vierergruppe.html}{Vierergruppe/Klein four-group}
        \item The \textbf{kernel} $K$ of group $G$ is the set of all $k$ such that $\Phi(k)=I_H$ (kinda a ``nullspace'' for groups)
        \item For a homomorphic map $\Phi:G\mapsto H$ between two groups $(G,*)$, $(H,\times)$,\begin{itemize}
            \item $\Phi(g_1*g_2)=\Phi(g_1)\times\Phi(g_2)$
            \item (Identity maps to identity) $\Phi(I_G)=I_H$
            \item (Inverses maps to inverses) $\Phi(g^{-1})=\left[\Phi(g)\right]^{-1}$
            \item Example: $(\mathbb{R},+)$ and $(U(1),\times)$, where $U(x)=\{c\in\mathbb{C}\ |\ |c|=1\}$, $\Phi(x)=e^{ix}$, $\Phi(x+2\pi)=\Phi(x)$ so only surjective, not bijective. Kernel $K=\{2n\pi\ |\ n\in\mathbb{Z}\}$
        \end{itemize}
        \item \fbox{The kernel is a \textbf{normal} subgroup} of $G$ because\begin{itemize}
                \item It is closed, $\forall k_1,k_2\in K,\ k_1*k_2\in K$
                \item $I_G\in K$ (identity maps to identity)
                \item inverse exists, $\forall k\in K,\ k^{-1}\in K$
                \item $\forall k\in K,\ \Phi(gkg^{-1})=\Phi(g)\Phi(k)\Phi(g^{-1})=\Phi(g)I_H[\Phi(g)]^{-1}=I_H\implies gkg^{-1}\in K$
            \end{itemize}
        \item The \textbf{product} of two cosets is the set of all products of two elements from each set $C_1\times C_2=\{c_1c_2\ |\ c_1\in C_1,c_2\in C_2\}$
        \item The \textbf{direct product} of two groups is similar but forms ordered pairs. $G\times H=\{(g,h)\ |\ g\in G,h\in H\}$\newline
                And $(g_1,h_1)\cdot(g_2,h_2)=(g_1*g_2,h_2\times h_2)$\newline
                Example: $D_4 = C_2\times C_2$ ($D_4$ is Klein four group, $C_2$ is 2nd order cyclic group)
        \item For a \underline{normal subgroup} $K$ of $G$, product of cosets $|(g_1K)(g_2K)|=|K|\neq|K|^2$ (not all distinct)\newline
                For a $g\in(g_1K)(g_2K)$, $g=g_1k_1g_2k_2=g_1(g_2g_2^{-1})k_1g_2k_2=g_1g_2k_1'k_2 = g_1g_2k_3\in g_1g_2K$
        \item For a cosets of non-normal groups, $|(g_1K)(g_2K)|\neq|K|$
        \item A \textbf{quotient group}, $G/K$, for a \underline{normal subgroup} $K$ is a \textcolor{red}{group of its cosets} in $G$.\newline
                Example: for $D_3$, $H=\{e,r,r^2\}$, $D_3/H=\{H, sH\}$, because $H\cdot H=H, H\cdot sH=sH$, $sH\cdot sH=H$, it's isomorphic to $C_2$
        \item (\textbf{Factorization theorem})(\href{https://math.libretexts.org/Bookshelves/Abstract_and_Geometric_Algebra/First-Semester_Abstract_Algebra%3A_A_Structural_Approach_(Sklar)/09%3A_The_Isomorphism_Theorem/9.01%3A_The_First_Isomorphism_Theorem}{Proof})\newline
                \fbox{If $K$ is the kernel of a homomorphism $\Phi:G\mapsto H$, then $G/K$ is isomorphic to $H$.}
        \item \textbf{Cayley's theorem}: every order-$N$ finite group is isomorphic to a subgroup of $S_n$ ($S_n$ reminds me of power sets)
        \item $S_{n-1}$ is a subgroup of $S_n$ (fix one item)
        \item An $n$-cycle in $S_n$ is a permutation that acts only on the positions $p_r,r=1,2,\ldots,n<N$ written as $(p_1,p_2,\ldots,p_n)$ which stands for $\begin{pmatrix}
                p_1 & p_2 & \cdots & p_n\\
                p_2 & p_3 & \cdots & p_1
            \end{pmatrix}$
        \item Any permutation can be decomposed uniquely into disjoint cycles.\newline
            \includegraphics*[width=0.6\linewidth]{permutation_example.png}
        \item A two cycle is a transposition/swap. $S_2=C_2$
        \item A permutation is odd/even if it is a product of odd/even permutations. An $n$-cycle can be decomposed into $(n-1)$ 2-cycles (decomposition \emph{not unique, not disjoint} in general). $n$ cycles's parity depends on $n-1$.
        \item The \textbf{cycle shape} of an element $S_n$ is the set of numbers $\{n_2,n_3,\ldots\}$ specifying the number of 2-cycles, 3-cycles,$\ldots$ in the \textbf{unique} decomposition into \textbf{disjoint} cycles.
        \item The disjoint cycles are the conjugacy classes of $S_n$.
        \item $G$ is the direct product of $H$ and $J$ ($G=J\times H=H\times J$) if\begin{itemize}
                \item $H$ and $J$ are \textbf{normal subgroups} of $G$
                \item $H$ and $J$ are \textbf{disjoint}, apart from the identity
                \item $G$ is \textbf{generated only by} elements of $H$ and $J$
            \end{itemize}
        \item If $G=H\times J$, $G/H$ is isomorphic to $J$\newline
                $G/H=\{jH\ |\ j\in J\}$, $\Phi:j\mapsto jH$ is a homomorphism because $H$ is normal ($j_1j_2H=(j_1H)(j_2H)$)\newline
                $j_1\neq j_2\implies j_1H\neq j_2H$, $\Phi$ is 1-1 so also isomorphic.
    \end{enumerate}

\section*{Representations}
    \begin{enumerate}
        \item A n-dimensional \textbf{representation} of a group $G$ is a homomorphism from $G$ to a subgroup of $GL(n,\mathbb{C})$. It's \textbf{faithful} if it's also isomorphic, otherwise \textbf{unfaithful}.
        \item \textbf{Regular representation} $G$ are matrices $D(g)$ generated by\\ $G=\{g_1=e,g_2,\ldots,g_N\}$, $\textcolor{red}{g^{-1}}G=\{g_{p(1)},g_{p(2)},\ldots,g_{p(N)}\}$,
                $D_{ij}=\delta_{p(i),j}=\delta_{i,p^{-1}(j)}$\\
                (Row order of $D(g)$ is element order of $g^{-1}G$, $D(e)=\vb{I}$)
        \item Two sets of matices are \textbf{equivalent} if they are similar by an invertible matrix $S$ (similarity transformation)
        \item If complex conjugate matrix $D^*(g)\sim D(g)$ for all $g$, can you make $D(g)$ entirely real? It's unknown, if you can, it's real. Otherwise, $D$ is pseudo-real.
        \item \textbf{Direct sum} of representations $D^{(1)}(g)\oplus D^{(2)}(g)=\begin{pmatrix}
                    D^{(1)}(g) & 0\\0 & D^{(2)}(g)
                \end{pmatrix}$ is a block-diagonal matrix.
        \item \textbf{Direct product} of representations $\otimes$, equations are cumbersome, better just see its picture.
        \item The \textbf{character} of a representation is $\{\Tr(D(g))\ |\ g\in G\}$\begin{itemize}
                \item Due to property of trace, equivalent representation have identical character
                \item $g$ in the same \emph{conjugacy class} have the same character
                \item Character of regular representation $\{|G|,0,0,\ldots,0\}$
            \end{itemize}
        \item Invariant subspace of a set of linear operations $\{A_i\}$ is\\ $W\subseteq V$ such that $x\in W, \vb{A}_ix\in W$, $\forall A_i\in \{A_i\}$.\\
            \begin{remark}
                higher dimensional ``eigenvector''-like stuff
            \end{remark}
        \item A representation $\{D(g)\}$ is irreducible representation (\textbf{irrep}) iff $\{0\}$ and $V=\mathbb{C}^n$ are the only invariant subspaces, where dimension of $D(g)$ is $n$.
        \item A 2D non-abelian representation is irreducible (diagonal matrices are abelian)
        \item The group-invariant inner product $[\vb{x},\vb{y}]=\sum_{g\in G}(\vb{D}(g)\vb{x},\vb{D}(g)\vb{y})$, where $(\vb{x},\vb{y})=\vb{x}^\dagger\vb{y}$\begin{itemize}
                \item $(\vb{D}(h)\vb{x},\vb{D}(h)\vb{y})=[\vb{x},\vb{y}]$, $h\in G$
                \item Length is $[\vb{x},\vb{x}]^{1/2}$
                \item Group representations $D(g)$ \textbf{unitary} using this inner product (because length preserved), $GL(n,\mathbb{C})$ similar to $U(n)$
                \item Unitary matrices have orthogonal eigenvectors, thus ``mutually orthogonal invariant subspaces''
                \item The representations of finite groups can always be taken to be unitary with respect to this inner product, if we use \textbf{orthonormal basis}.
            \end{itemize}
        \item All 1D (unfaithful) represetations are irreducible (no subspace that's not $\{0\}$ or itself). Any two different 1D representations are inequivalent.
        \item Character table of \textbf{irreps} (irreducible representations) of $D_4$\\
            \begin{center}
                \includegraphics*[width=0.5\linewidth]{character_table D_4.png}
            \end{center}
            \begin{itemize}
                \item \textcolor{red}{$\chi_i^J$} denotes $i$-th column, $J$-th row in the character table above
                \item First four unfaithful 1D, last one faithful 2D
                \item Vertical lines separates conjugacy classes
                \item Column in different conjugacy classes are orthogonal
                \item Rows are orthogonal (gridlock orthogonality)
                \item rows sum to 0, except for the 1st one \mylabel{thm:row_sum}{$\bigstar$}
                \item (\textbf{Theorem 1}) The number of inequivalent irreps $\rho$ = the number of \textbf{conjugacy classes} $c$ \mylabel{thm:irrep1}{$\bigstar$}\\
                    (Horizontal, vertical, both 5 cells, counting each conjugacy class as one)
                \item (\textbf{Theorem 2}) \\Sum of \textbf{squared} dimensions of inequivalent irreps in a conjugacy class = $|G| = \sum_{J=1}^\rho d_{J}^2$ \mylabel{thm:irrep2}{$\bigstar$}\\
                    (only last/fifth representation 2D, others 1D. $1^2+1^2+1^2+1^2+2^2=8$)\\
                    A special case of column orthogonality.
                \item (\textbf{Theorem 3}) The dimension $d_J$ of each irrep \textbf{divides} $|G|$ \mylabel{thm:irrep3}{$\bigstar$}
                \item (Corollary of 2) Every irrep has dimension $\leq (|G|-1)^{1/2}$, \\
                        groups with $|G|<5$ cannot have have 2d irrep, they're abelian. \\
                        Only $|C_5|=5$ because 5 is prime, \\
                        so smallet non-abelian group is $D_3$, $|D_3|=6$.
            \end{itemize}
        \item (\textbf{Schur's 1st lemma}) For two irreps with matrices $D_1(g):U\rightarrow U$, $D_2(g):V\rightarrow V$, define the \emph{intertwining operator} $T:U\rightarrow V$ to be 
                \[
                    \forall g\in G,\quad TD_1(g)=D_2(g)T,
                \] 
            then \begin{itemize}
                \item $T=\vb{0}$. The irreps can be equivalent or inequivalent.
                \item $T$ is invertible, and the irreps must be equivalent. $U=V$ have same dimensions\mylabel{lma:schur1}{$\bigstar$}
            \end{itemize}
        \item (\textbf{Schur's 2nd lemma}) A matrix that commutes with all $D(g)$ of an irrep is $T=\lambda\vb{I},\ \lambda\in\mathbb{C}$\mylabel{lma:schur2}{$\bigstar$}
        \item (\textbf{The grand orthogonality theorem}) Let $D^J(g):G\rightarrow V_J$ be matrices of an irrep, $V_J=\text{GL}_d(\mathbb{C})$, $J$ loops through all \textbf{inequivalent} irreps, \textcolor{red}{$d=\dim V_J$}
                \[
                    \frac{1}{|G|}\sum_{g\in G}(D_{ij}^J(g^{-1})) D_{kl}^K(g)=\frac{1}{d}\delta_{jk}\delta_{il}\delta^{JK}
                \]\where{$\delta^{JJ}=1$ is \textcolor{red}{not summed over} \mylabel{thm:grand_orthogonality}{$\bigstar$}}
        \item \[
                \boxed{\sum_{g\in G}D_{ij}^\dagger(g)D_{kl}(g)=\frac{|G|}{d}\delta_{il}\delta_{jk}}
            \]
        \item $g$ in same conjugacy class have the same trace/character\\ so character $\chi(g)=\Tr D(g)$ is a \emph{class function}.
        \item (\textbf{Row orthogonality}) Take trace on boths sides ($i=j,k=l$)\[
            \sum_{g\in G}\left(\chi^J(g)\right)^*\chi^K(g) = \sum_{i=1}^c |C_i|(\chi^J_i)^*\chi^K_i =|G|\delta^{JK}=\begin{cases}
                                                                                                                        |G|,& \qif i=j\\
                                                                                                                        0,& \qif i\neq j
                                                                                                                    \end{cases}
            \]\where{$|C_i|$ is size of $C_i$,\\
                $\chi_i$ is $\chi(g)$ for $g$ in $C_i$,\\
                $C_i$ is the $i$-th conjugacy class}
            \begin{remark}
                $\vb{\chi}$ are $c$-dimensional vectors, for $\rho$ inequivalent irreps, we have $\rho$ distinct $\chi$. \textcolor{red}{$\rho\leq c$} for linear independence
            \end{remark}
        \item (\textbf{Column orthogonality}) \mylabel{thm:col_orthog}{$\bigstar$}\[
                \sum_{J=1}^\rho(\chi_i^J)^*\chi_j^J=\begin{cases}
                    \frac{|G|}{|C_i|},&\quad \qif i=j\\
                    0,&\quad \qif i\neq j
                \end{cases}
            \]
            \begin{remark}
                $\vb{\chi}$ are $\rho$ dimensional vectors, we have $c$ distinct such vectors, \textcolor{red}{$c\leq\rho$}.
            \end{remark}
        \item \myref{thm:irrep1} $\rho=c$ because of column and row orthogonality, $c\leq\rho$ and $\rho\leq c$.
        \item $\chi_1^J=d_J$; for each irrep $J$, $\chi_1=\chi(e)=Tr(I^{(d_J)})=d_J$ is the dimension of the representation.
        \item \myref{thm:irrep2} From column orthogonality, the first column (identity $e$), $\sum_{J=1}^\rho|\chi_1^J|^2=\sum_{J=1}^\rho d_J^2 = \frac{|G|}{|C_1|}$, $|C_1|$ is the size of first conjugacy class, $\{e\}$, $|C_1|=1$.
        \item $\chi(D_1(g)\oplus D_2(g))=\chi(D_1(g))+\chi(D_1(g))$,\\ $\chi(D_1(g)\otimes D_2(g))=\chi(D_1(g))\chi(D_1(g))$
        \item (\textbf{Decomposition} of a reducible representation) \[\vb{S}\vb{D}(g)\vb{S}^{-1}=\bigotimes_{J=1}^\rho \left(\vb{I}^{(n_J)}\otimes\vb{D}^J(g)\right)\]
                \where{$\vb{I}^{(n)}$ is $n\times n$ identity matrix,\\ $J$ loops through $\rho$ inequivalent irreps,\\ $n_J$ is number of times $J$-th irrep $\vb{D}^J(g)$ occured}
        \item Take trace on both sides, $\chi(g)=\sum_{J=1}^\rho n_J\chi^J(g)$
        \item $\boxed{n_J = \langle\chi,\chi^J\rangle = \frac{1}{|G|}\sum_{i}^c |C_i|\chi(g)(\chi^J(g))^* = \frac{1}{|G|}\sum_{g\in G}\chi(g)^*\chi^J(g)}$ is always an \textcolor{red}{integer} (multiplicity of $\chi^J$ in $\chi$)
        \item \myref{thm:row_sum} Let $\chi^1$ be the trivial irrep (all 1s, so $\chi^1(g)=1$), $\langle\chi^J,\chi^1\rangle=\frac{1}{|G|}\sum_{g\in G}\chi^J(g)=\delta^{J1} \implies \sum_{g\in G}\chi^J(g) = |G|\delta^{J1}$
    \end{enumerate}

    \section*{Other tricks}
    \begin{enumerate}
        \item Remove poles/infinities by introducing a small value, and set it to 0 after all the calculations
        Examples\begin{itemize}
            \item (Lecture notes) Evaluate $\frac{1}{2\pi}\int_{-\infty}^{\infty} e^{ikx} \dd k$ by considering $\frac{1}{2\pi}\int_{-\infty}^{\infty} e^{ikx-\epsilon|k|} \dd k$
            \item (Lecture notes) Prove $\mathcal{F}[H(x-a)]=\frac{e^{-ika}}{\sqrt{2\pi}ik}$ with a similar technique. (Also relate $\mathcal{F}[f]$ with $\mathcal{F}[f']$)
            \item (Problem sheet 1)
            
            \includegraphics*[width=0.8\linewidth]{Mawell_Ampere_Stokes.png}
            \item (Wikipedia) Sturm-Liouville operator $L=\dv{x}\left[p(x)\dv{x}\right]+q(x)$
            \item (Wikipedia) Boundary condition operator $\vb{D}u=\begin{pmatrix}
                \eval{A_1\dv{x}+B_1}_{x=0}\\
                \eval{A_2\dv{x}+B_2}_{x=l}
            \end{pmatrix}$
            \item (Wikipedia) d'Alembert operator $\Box = \partial^\mu\partial_\mu=\eta^{\mu\nu}\partial_\nu\partial_\mu = \frac{1}{c^2}\pdv[2]{t}-\laplacian$
        \end{itemize}
    \end{enumerate}

    \section*{Examples}
    \begin{enumerate}
        \item $\int_{-\infty}^{+\infty}\frac{\sin^2x}{x^2}\dd x=\pi$, \href{https://math.stackexchange.com/questions/141695/how-to-calculate-the-integral-of-sin2x-x2}{answer}, \href{https://en.wikipedia.org/wiki/Lobachevsky_integral_formula}{a theorem}
        \item (hint: Feymann's trick) $\int_{-\infty}^{\infty}\frac{x^2e^x}{(e^x+1)^2}\dd x=\pi^2/3$, \href{https://math.stackexchange.com/questions/1801106/how-does-one-integrate-x2-fracexex12}{answer}
        \item Scale solution $\Phi$ of $\laplacian{\Phi}=0$ to fit similar boundary conditions: $\Phi\left(x,\frac{y}{a}\right)$ is wrong. $\Phi\left(\frac{x}{a},\frac{y}{a}\right)$ is correct (Lent question sheet 4, Q1)
    \end{enumerate}

    \section*{Proofs}
    \begin{enumerate}
        \item \myref{greens}\begin{align*}
            1 = \lim_{x\rightarrow0}\int_{\zeta-\epsilon}^{\zeta+\epsilon}\delta(x-\zeta)\dd x &= \lim_{x\rightarrow0}\int_{\zeta-\epsilon}^{\zeta+\epsilon}\mathcal{L}G(x,\zeta)\dd x\\
            &= \lim_{x\rightarrow0}\int_{\zeta-\epsilon}^{\zeta+\epsilon}\left(\pdv[2]{G}{x}+p(x)\pdv{G}{x}+qG\right)\dd x\\
            &= \lim_{x\rightarrow0}\int_{\zeta-\epsilon}^{\zeta+\epsilon}\pdv{x}\left(\pdv{G}{x}+pG\right) + \left(-\pdv{p}{x}+q\right)G\dd x\\
            &= \lim_{x\rightarrow0}\int_{\zeta-\epsilon}^{\zeta+\epsilon}\pdv{x}\left(\pdv{G}{x}+pG\right)\dd x\\
            &= \lim_{x\rightarrow0}\left[\pdv{G}{x}+pG\right]_{\zeta-\epsilon}^{\zeta+\epsilon}
        \end{align*} \wher{because $p,q$ are continuous}
        Thus $G$ is continuous, satisfies the same boundary condition as $y$ and $\lim_{x\rightarrow0}\left[\pdv{G}{x}\right]_{\zeta-\epsilon}^{\zeta+\epsilon}=1$.
        \item Schur's first lemma \myref{lma:schur1}\begin{itemize}
                \item Show $\mathrm{Kernel}(T)\subseteq U$ and $\mathrm{Image}(T)\subseteq V$ are invariant subspaces of $\{D_1(g)\ |\ g\in G\}$ and $\{D_2(g)\ |\ g\in G\}$ by using definitions on elements in them
                \item If $T\neq\vb{0}$, $\mathrm{Kernel}(T)=\{0\}$, $\mathrm{Image}(T)=V$, so $T$ is one-to-one and onto, invertible
                \item $D_1(g)=T^{-1}D_2(g)T$, the two irreps are equivalent
            \end{itemize}
        \item Schur's second lemma \myref{lma:schur2}\begin{itemize}
                \item $D(g)T=TD(g)$
                \item whether $T$ is singular or not, $T-\lambda I$ is singular, where $\lambda$ is one of its eigenvalues
                \item $D(g)T-\lambda D(g)=TD(g)-\lambda D(g)$, $D(g)(T-\lambda I)=(T-\lambda I)D(g)$
                \item use first lemma, $T-\lambda I$ is invertible or $0$
                \item $T-\lambda I=0$, $T=\lambda I$
                \item Better explanantion \href{https://sites.ualberta.ca/~vbouchar/MAPH464/section-schur.html}{here}
            \end{itemize}
        \item Grand orthogonality theorem \myref{thm:grand_orthogonality}\begin{itemize}
                \item Let $d=\dim V_J$, $M$ be any $d\times d$ matrix, $T=\sum_{g\in G}D^J(g^{-1})MD^K(g)$
                \item $gG=G$, element on whole group causes a derangement, $D^J(g)T=T=TD^K(g)$
                \item If $J=K$, use Schur's second lemma, $T=\lambda(M)\vb{I}\delta^{JK}$
                \item $\sum_{g\in G}D^J_{ij}(g^{-1})M_{jk}D^K_{kl}(g)=\lambda(M)\delta_{il}\delta^{JK}$, $\lambda$ depends on $M$
                \item $\sum_{g\in G}D^J_{ij}(g^{-1})D^K_{kl}(g)=\lambda_{jk}\delta_{il}\delta^{JK}$, where $\lambda_{jk}=\lambda(M^{jk})$, $M^{jk}$ has 1 at $M_{jk}$, otherwise 0
                \item $\lambda_{jk}\delta_{ii}\delta^{JJ}=\sum_{g\in G}D^J_{ij}(g^{-1})D^J_{ki}(g)=\sum_{g\in G}D_{ij}^J(g^{-1})D_{jk}^J(g)D_{jk}^J(g^{-1})D^J_{ki}(g)\\\null\qquad\qquad=\sum_{g\in G}D^J_{ik}(e)D^J_{ki}(e)=\sum_{g\in G}D_{jk}(e)=\sum_{g\in G}\delta_{jk}=|G|\delta_{jk}$\\
                $\delta_{ii}=d$, $\delta^{JJ}=1$ because $J$ is not summed over
                \item $\lambda_{jk}=\frac{1}{d}|G|\delta_{jk}$
                \item $D(g^{-1})=(D(g))^{-1}=(D(g))^\dagger$ because $D$ is unitary wrt. group invariant inner product
                \item $ \frac{1}{|G|}\sum_{g\in G}(D_{ij}^J(g^{-1}))^\dagger D_{kl}^K(g)=\frac{1}{d}\delta_{jk}\delta_{il}\delta^{JK}$
            \end{itemize}
        \item Column orthogonality of character table \myref{thm:col_orthog}\begin{itemize}
                \item Not obvious but \href{https://people.brandeis.edu/~igusa/Math101bS07/Math101b_notesD2e.pdf}{here we are}
                \item $\langle\chi_i^J,\chi_i^K\rangle=\frac{1}{|G|}\sum_{i=1}^c|C_i|(\chi_i^J)\chi_i^K=\delta^{JK}$
                \item Define character table matrix $\vb{X}$ using $\vb{X}_{iJ}=\chi^J_i$ (subscript horizontal/$G$, superscript vertical/$C_i$)
                \item $\vb{I}=\vb{X}\frac{1}{|G|}\begin{pmatrix}
                        |C_1| & \cdots & 0\\
                        0 & \ddots & 0\\
                        0 & \cdots & |C_c|
                    \end{pmatrix}\vb{X}^T=\vb{X}\vb{D}\vb{X}^T$
                \item $\vb{X}$ is $\rho\times c$, $\vb{D}$ is $c\times c$, so it fits. $\rho\neq c$ in general, yet.
                \item $\vb{X}\vb{D}\vb{X}^T=I, \vb{X}^T=\vb{D}^{-1}\vb{X}^{-1}, \vb{X}^{T}\vb{X}=\vb{D}^{-1}=|G|\begin{pmatrix}
                        \frac{1}{|C_1|} & \cdots & 0\\
                        0 & \ddots & 0\\
                        0 & \cdots & \frac{1}{|C_c|}
                    \end{pmatrix}$
                \item Products of columns of $\vb{X}$ goes to diagonal
                \item take $(\cdot)_{ij}$ on both sides, $\sum_{J=1}^\rho \chi^J_i\chi^J_j = \begin{cases}
                        \frac{|G|}{|C_i|} \qor \frac{|G|}{|C_j|},&\quad \qif i=j\\
                        0,&\quad \qif i\neq j
                    \end{cases}$
            \end{itemize}
        \item \myref{thm:irrep3} Proof \href{https://math.stackexchange.com/questions/243221/proofs-that-the-degree-of-an-irrep-divides-the-order-of-a-group}{here}\begin{itemize}
                \item Because I just want to sleep, see link above
                \item I would rather just start over everything reading abstract algebra in the future
            \end{itemize}
    \end{enumerate}

\end{document}