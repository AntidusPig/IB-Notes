\documentclass{article}
\usepackage{amsmath, amssymb, amsthm, amsfonts, bm}
\theoremstyle{remark}
\newtheorem*{theorem}{Theorem}
\newtheorem*{remark}{Remark}
\newtheorem*{definition}{Definition}
\newtheorem*{hypothesis}{Hypothesis}
\newtheorem*{corollary}{Corollary}
\theoremstyle{remark}

\usepackage{physics}
\usepackage[a4paper, total={6in,10in}]{geometry}
\usepackage[dvipsnames]{xcolor}
\usepackage{xcolor-material}
\usepackage{hyperref}
    \hypersetup{colorlinks=true, linkcolor=ForestGreen}
\usepackage{graphicx}
    \graphicspath{{./img/}}
\usepackage{tikz}
\usepackage{ragged2e}
\usepackage{array}   % for \newcolumntype macro
\newcolumntype{L}{>{$}c<{$}} % math-mode version of "l" column type

\usepackage{soul}

\newcommand{\where}[1]{\begin{flushright}where #1.\end{flushright}}
\newcommand{\wher}[1]{\begin{flushright}#1.\end{flushright}}
\newcommand{\mylabel}[2]{\hyperref[#1]{#2}\label{back:#1}}
\newcommand{\myref}[1]{\hyperref[back:#1]{$\bigstar$}\label{#1}}
\newcommand{\e}{\hat{\vb{e}}}  % unit vector
\newcommand{\s}[1]{\textsubscript{#1}}
\everymath{\displaystyle}
\begin{document}

\begin{enumerate}
    \item $\boxed{r=\frac{r_0}{1+e\cos\phi}}$
    \item $\boxed{J^2 = Amr_0}$
    \item Has a different origin to $\frac{x^2}{a^2}+\frac{y^2}{b^2}=1$, $\frac{x^2}{a^2}-\frac{y^2}{b^2}=1$ but same shape
    \item $r_{\mathrm{min}}=\frac{r_0}{1+e}$, $r_{\mathrm{max}}=\frac{r_0}{1-e}$
    \item $F=-\frac{A}{r^2}$
    \item Kepler\begin{itemize}
        \item 1st Trajectory are ellipses
        \item 2nd $\dv{\mathrm{Area}}{t}=r^2\dot\phi/2=\frac{J}{2m}$ is constant
        \item 3rd $T^2 \propto a^3$ (Rearrange polar to cartesian, semi-major $a=\frac{r_0}{1-e^2}$, semi-minor $b=\frac{r_0}{\sqrt{1-e^2}}$, $T=\frac{\pi ab}{\mathrm{area\ change\ rate}}=2\pi\sqrt{\frac{ma^3}{A}}$)
    \end{itemize}
    \item A cool thing I wasn't aware of is that $\dv{t}(r^2) = 2r\dv{r}{t} = \dv{t}(\vec{r}\cdot\vec{r}) = 2\vec{r}\cdot\dv{\vec{r}}{t}$, which is not immediately obvious
    
    $\dv{r}{t} = \hat{r}\cdot\dv{\vec{r}}{t}$ (draw a picture)

    \item $\vb{J}=\sum\vb{r}\times\vb{p}=\sum\vb{r}\times m(\bm{\omega}\times\vb{r})=\textcolor{red}{\sum m(r^2\bm{\omega}-\vb{r}(\vb{r}\cdot\bm{\omega}))} = \sum m(\bm{r}^T\bm{r}\bm{1}-\bm{r}\bm{r}^T)\bm{\omega} = \bm{I}\bm{\omega} = \begin{pmatrix}
        \textstyle\sum m(y^2+z^2) & -\textstyle\sum mxy & -\textstyle\sum mxz\\
        -\textstyle\sum mxy & \textstyle\sum m(x^2+z^2) & -\textstyle\sum myz\\
        -\textstyle\sum mxz & -\textstyle\sum myz & \textstyle\sum m(x^2+y^2)\\
    \end{pmatrix}\bm{\omega} $

    ($I$ is Hermitian, principle axes orthogonal)

    \item $T = \frac{1}{2}\sum m(\bm{\omega}\times\bm{r})\cdot(\bm{\omega}\times\bm{r}) = \frac{1}{2}\sum m\bm{\omega}\cdot(\bm{r}\times(\bm{\omega}\times\bm{r})) = \frac{1}{2}\bm{\omega}\cdot\bm{J} = \frac{1}{2}\bm{\omega}^T I\bm{\omega}$
    
    The surface of constant $T$ is a quadric surface called Inertia Ellipsoid. $\grad_{\bm{\omega}}T=\bm{J}$, $\bm{J}$ is perpendicular to the surface of constant $T$

    \item Perpendicular axes theorem for sheets, parallel axes theorem $I=I_0+Ma^2$ for $I_0$ at CoM at $\bm{a}$ from the origin
    \item Kater's pendulum, parallel axes theorem + pendulum = determine $g$ by measuring small oscillation period $T$ and $a$
    \item For body frame $S$, $\bm{G}=\left[\dv{\bm{J}}{t}\right]_S+\bm{\omega}\times\bm{J}$, Euler's equations are \begin{align*}
        G_1 = I_1\dot\omega_1 + (I_3-I_2)\omega_2\omega_3\\
        G_2 = I_2\dot\omega_2 + (I_1-I_3)\omega_3\omega_1\\
        G_3 = I_3\dot\omega_3 + (I_2-I_1)\omega_1\omega_2
    \end{align*}
    
    (Because $\bm{J}=I\bm{\omega} = \sum_{i=1}^{3} I_i\omega_i\hat{e}_i$, $\bm{G}=\sum_{i=1}^{3} I_i\dot\omega_i\hat{e}_i+I_i\omega_i\dv{\hat{e}_i}{t}$ and $\boxed{\dv{\hat{e}_i}{t}=\bm{\omega}\times\hat{e}_i}$)
    \item For a symmetric top ($I_1=I_2\neq I_3$), body freqency $\boxed{\Omega_b\equiv\frac{I_1-I_3}{I_1}\omega_3}$\begin{itemize}
        \item In the body frame $S$, $\bm{J}$ and $\bm{\omega}$ in the same plane because $I_1=I_2$, and if oblate inertia ellipsoid (prolate top), $I_3>I_2$, $J_3=I_3\omega_3>I_1\omega_3$, $\bm{J}$ inside the cone of $\bm{\omega}$
        \begin{center}
            \includegraphics[width=0.25\linewidth]{symmetric top body frame.png}
        \end{center}
        \item In the inertial frame $S_0$, rate of precession $\boxed{\Omega_s=\frac{\dot\omega_1}{|\bm{\omega}|\sin\theta_s}=\frac{J}{I_1}}$
        \begin{center}
            \includegraphics[width=0.6\linewidth]{rate of precession.png}
        \end{center}
        \item Ellipsoid tangential to invariable plane ($\grad_{\bm{\omega}}T=\bm{J}$), and rolls without slipping on it. 
        \[\boxed{\Omega_b\sin\theta_b=\Omega_s\sin\theta_s}\]
        \begin{center}
            \includegraphics[width=0.6\linewidth]{poinsot.png}
        \end{center}
    \end{itemize}
    \item For triaxial body with $I_1<I_2<I_3$, if the body spins about the 2-axis is unstable. $\bm{\omega}$ can change while keeping $\bm{J}$ and energy constant.
    \item Major axis theorem: non-rigid bodies will align their $\bm{J}$ to the major axis to minimize energy
    \item Symmetric top with Euler angles $(\theta,\phi,\chi)$
    
    \begin{minipage}[b]{0.6\textwidth}
        \centering
        \begin{itemize}
            \item $\bm{\omega}=\dot\phi\hat{e}_z+\dot\theta\hat{e}_1+\dot\chi\hat{e}_3$
            \item In body frame $S$, $\bm{\omega}=(\dot\theta,\dot\phi\sin\theta,\dot\chi+\dot\phi\cos\theta)$
            
            $\bm{J}=(I_1\dot\theta,I_1\dot\phi\sin\theta,I_3(\dot\chi+\dot\phi\cos\theta))$
            \item Keep $\omega_3=\dot\chi+\dot\phi\cos\theta$, $J_z=J_3\cos\theta+J_2\sin\theta$ constant
            \item We get $\dot\phi=\Omega_s$, $\dot\chi=\Omega_b$
        \end{itemize}
    \end{minipage}
    \hfill
    \begin{minipage}[b]{0.39\textwidth}
        \centering
        \includegraphics[width=0.4\linewidth]{Lagrange's approach.png}
    \end{minipage}

    \item Equation of motion with gravity and support
    \begin{align*}
        E &= \frac{1}{2}I_1\omega_1^2 + \frac{1}{2}I_1\omega_2^2 + \frac{1}{2}I_3\omega_3^2+mgh\cos\theta\\
          &= \frac{1}{2}I_1\dot\theta^2 + \frac{J_2^2}{2I_1} + \frac{J_3^2}{2I_3}+mgh\cos\theta\\
          &= \frac{1}{2}I_1\dot\theta^2 + \frac{(J_z-J_3\cos\theta)^2}{2I_1\sin^2\theta} + \frac{J_3^2}{2I_3}+mgh\cos\theta\\
          &= \frac{1}{2}I_1\dot\theta^2 + U_{\mathrm{eff}}(\theta)
    \end{align*}

    \item Sleeping top $J_z=J_3$; if $\dv{U_{\mathrm{eff}}}{\theta}=0$ steady precession; oscillation around $\theta$ is nutation
    \begin{center}
        \includegraphics*[width=0.3\linewidth]{symmetric top effective potential.png}
    \end{center}

    \item $\mathcal{L}=T-V$, $\dv{t}\pdv{\mathcal{L}}{\dot q_i}=\pdv{\mathcal{L}}{q_i}$
    \item Conjugate momenta $p_i=\pdv{L}{\dot q}$, symmetry (\textbf{invariance} of $\mathcal{L}$ wrt. $q_i$ leads to \textbf{conservation} of $p_i$)
    \item Hamiltonian $H(q_i,p_i,t)\equiv\sum_i p_i\dot q_i-\mathcal{L}(q_i,\dot q_i,t)$, 
    $\dd H=\sum_i\left(\dot q_i\dd p_i-\dot p_i\dd q_i\right)-\pdv{\mathcal{L}}{t}\dd t \allowbreak= \sum_i\left(\pdv{H}{q_i}\dd q_i+\pdv{H}{p_i}\dd p_i\right)+\pdv{H}{t}\dd t \allowbreak= -\pdv{\mathcal{L}}{t}$ (depends only on $q_i,p_i,t$, not $\dot q_i$)
    \item $\pdv{H}{t}=-\pdv{\mathcal{L}}{t}$ means if $\mathcal{L}$ is independent of $t$, then energy/Hamiltonian is conserved
    \item $\dot q_i=\pdv{H}{p_i}$, $\dot p_i=-\pdv{H}{q_i}$
    \item $\epsilon=x-x_0$, $m\ddot x+\dv{U}{x}=0$, $m\ddot\epsilon+U_0''\epsilon=0$
    \item In a \textbf{normal mode} every element of the system oscillates at a single frequency, a general free oscillation of the system can be expressed in terms of a linear combination of the single normal modes.
    \item $\vb{r}=\vb{r}(\{q_i\})$, around equilibrium $\bm{\dot r}\approx\sum_i\dot q_i\eval{\pdv{\bm{r}}{q_i}}_{\mathrm{eq}}$, $T=\frac{1}{2}\sum_k m_k|\bm{\dot r}_k|^2 = \frac{1}{2}\sum_{ij}M_{ij}\dot q_i\dot q_j = \frac{1}{2}\bm{\dot q}^T\bf{M}\bf{\dot q}$, $M_{ij}=\sum_k m_k\eval{\pdv{\bf{r_k}}{q_i}}_{\mathrm{eq}}\eval{\pdv{\bf{r_k}}{q_j}}_{\mathrm{eq}}$
    \item At equilibrium $\eval{\pdv{U}{q_i}}_{\mathrm{eq}}$, $U=U(\bm{q})\approx U_0+0+\frac{1}{2}\sum_{ij}q_i q_j\eval{\pdv{^2U}{q_i \partial q_j}}_{\mathrm{eq}}+\ldots$, $K_{ij}=\eval{\pdv{^2U}{q_i \partial q_j}}_{\mathrm{eq}}$
    \item At equilibrium $E\approx U_0+\frac{1}{2}\sum_{ij}M_{ij}\dot q_i\dot q_j+\frac{1}{2}\sum_{ij}K_{ij}q_i q_j$, $\dv{E}{t}=0=\sum_{ij}\dot q_i(M_{ij}\ddot q_j+K_{ij}q_j)$, $\boxed{\bm{M}\vb{\ddot q}+\bm{K}\vb{q}=0}$, together with guessed solution $\boxed{\vb{q}(t)=\vb{Q}e^{i\omega t}}$, $(\bm{K}-\omega^2\bm{M})q = 0$
    \item $\bm{M}$ and $\bm{K}$ are symmetric, thus $\omega_i$ are real
    \item \begin{itemize}
        \item $(\vb{K}-\omega^2 \vb{M})\vb{q}=0$, $\vb{K},\vb{M}$ are symmetric, $\omega^2$ is real
        \item $(\vb{M}^{-1}\vb{K}-\omega^2\vb{I})\vb{q}=0$, $\vb{M}^{-1}\vb{K}$ not symmetric in general, $\vb{q}_i\cdot\vb{q}_j\neq0$ in general (product of symmetric matrices may not be symmetric)
        \item $(\vb{M}^{-1/2}\vb{K}\vb{M}^{-1/2}-\omega^2\vb{I})(\vb{M}^{1/2}\vb{q})=0$, inverse and square root of symmetric matrices are symmetric, $\vb{M}^{-1/2}\vb{K}\vb{M}^{-1/2}$ is symmtric, $\boxed{\vb{q}_i^T\vb{M}\vb{q}_j = \delta_{ij}}$
        \item To find orthonormal modes, turn $\vb{M}$ into $\vb{I}$ and normalize $\vb{q}$
    \end{itemize}
    \item Young's modulus $E=\frac{P}{\delta l/l}$, Bulk modulus $B=-\frac{P}{\delta V/V}$ ($\Delta V<0$)
    \item $\dd\vb{F}=\bm{\tau}\dd\vb{S}=A\bm{\tau}\cdot\hat{\vb{n}}$, the stress tensor is $\bm{\tau}=\begin{pmatrix}
        \tau_{xx} & \tau_{xy} & \tau_{xz} \\
        \tau_{yx} & \tau_{yy} & \tau_{yz} \\
        \tau_{zx} & \tau_{zy} & \tau_{zz}
    \end{pmatrix}$, $\tau_{xx}$ is normal stress, $\tau_{xy}$ is shear stress
    \item $\vb{X}=\vb{e}\vb{x}$, the strain tensor is $\vb{e}=\begin{pmatrix}
        e_{xx} & e_{xy} & e_{xz}\\
        e_{yx} & e_{yy} & e_{yz}\\
        e_{zx} & e_{zy} & e_{zz}\\
    \end{pmatrix}$ is symmetric, $\boxed{e_{ij}=\frac{1}{2}\left(\pdv{X_i}{x_j}+\pdv{X_j}{x_i}\right)}$, can be diagonalized to $\vb{e}=\begin{pmatrix}
        e_1 & 0&0\\
        0 & e_2&0\\
        0 & 0&e_3\\
    \end{pmatrix}$
    \item For simplicity, only consider principal axis to express $e$ and $\tau$ as vectors
    \item Strain is $e=\delta l/l$, stress is $\tau=-P=-F/A$, for isotropic material, $\textstyle E\vb{e}=\begin{pmatrix}
        1 & -\sigma & -\sigma\\
        -\sigma & 1 & -\sigma\\
        -\sigma & -\sigma & 1
    \end{pmatrix}\bm{\tau}$, $\sigma$ is Poisson ratio, $e_1=e_2=e_3=\frac{\tau(1-2\sigma)}{E}$, $\frac{\delta V}{V}\approx e_1+e_2+e_3 = \frac{3\tau(1-2\sigma)}{E}$, $\boxed{B=\frac{E}{3(1-2\sigma)}}$
    \item $\bm{\tau}=\begin{pmatrix}
        1 & -\sigma & -\sigma\\-\sigma & 1 & -\sigma\\-\sigma & -\sigma & 1\end{pmatrix}^{-1}
        E\vb{e} = \frac{E}{(\sigma+1)(1-2\sigma)}\begin{pmatrix}
            1-\sigma & \sigma & \sigma\\
            \sigma & 1-\sigma & \sigma\\
            \sigma & \sigma & 1-\sigma\\
        \end{pmatrix}\vb{e} = \lambda(e_1+e_2+e_3)+2G\bm{e} = \lambda\Tr(\bm{e})\bm{I}+2G\bm{e}$, Lam\'e's constant $\boxed{\lambda\equiv\frac{E\sigma}{(1+\sigma)(1-2\sigma)}, G=\frac{E}{2(1+\sigma)}}$, $\lambda=B-\frac{2}{3}G$
    \item The elastic potential energy of a small volume $(\Delta x,\Delta y,\Delta z)$ is \begin{align*}
        U &=\frac{1}{2}\Delta x\Delta y\Delta z(\tau_1 e_1+\tau_2 e_2+\tau_3 e_3) \\
        &= \frac{1}{2}\Delta x\Delta y\Delta z\left[\lambda(e_1+e_2+e_3)^2+2G(e_1^2+e_2^2+e_3^2)\right] \\
        &= \frac{1}{2}\Delta x\Delta y\Delta z\Tr(\bm{\tau}\bm{e})\\ 
        &= \frac{1}{2}\Delta x\Delta y\Delta z(\tau_{xx}e_{xx}+\tau_{yy}e_{yy}+\tau_{zz}+e_{zz}+2\tau_{xy}e_{xy}+2\tau_{yz}e_{yz}+2\tau_{xz}e_{xz})\\
        &= \frac{1}{2}\Delta x\Delta y\Delta z(\Tr[(\lambda \Tr(\bm{e})+2G\bm{e})\bm{e}])\\
        &= \frac{1}{2}\Delta x\Delta y\Delta z(\Tr(\bm{e})\Tr[\lambda\bm{e}]+2G\Tr[\bm{e}^2])\\
        &= \textcolor{red}{\frac{1}{2}}\Delta x\Delta y\Delta z\ (\textcolor{red}{\lambda[\Tr(\bm{e})]^2+2G\Tr(\bm{e}^2)})
    \end{align*}
    \item For a bending beam, the bending moment (sum of moment caused by all forces at cross section) is $\boxed{B=\frac{EI}{R}}$, moment of area $\boxed{I=\int_{\text{cross section}} y^2\dd A}$\begin{align*}
        B&=\int y\cdot\text{stress}\dd A \\
         &=\int yE\cdot\text{strain}\dd A \\
         &=\int yE\cdot\frac{\Delta l}{l}\dd A\\
         &=\int yE\cdot\frac{\theta(R+y)-\theta R}{\theta R}\dd A\\
         &=\int yE\frac{y}{R}\dd A
    \end{align*}
    where radius of curvature $R\approx \frac{1}{y''}$, $\boxed{B=EIy''}$
    \item For a general beam with load per unit length $W(x)$, $W=-\dv{F}{x},\boxed{F=-\dv{B}{x}}$, $\boxed{W=EIy''''}$
    \item Finding out bending moment: the beam is at equilibrium, so the part to the left of $x$ is at equilibrium, the RHS tip of this part --- element at $x$, is balanced by its bending moment and all the forces on the left (for non-rigid/elastic body, balance of force is a necessary insufficient condition of equilibrium)
    
    Interestingly, using this analysis, a beam freely supported at one end only cannot be in equilibrium, as the part to the right of the load cannot have $B(x)=0$. It must be clamped on the left to provide a torque so the $B(x)$ is moved upwards to make that 0.
    \item $B=-Fy,y''+\frac{F}{EI}y=0, y=A\sin\sqrt{\frac{F}{EI}}x$, Euler force $F_E=\frac{\pi^2 EI}{L^2}$, of $F<F_E$ the beam is compressed, if $F\geq F_E$ the beam will bend suddenly. (Note that $L$ also changes)\begin{center}
        \includegraphics[width=0.24\linewidth]{euler_struct.png}
    \end{center}
    \item $\pdv{\rho}{t}+\div(\rho\vb{v}) = 0$, $\rho$ is a constant (incompressible), $\div\vb{v}=0$
    \item $\vb{v}(\vb{x},t)$, $\dd\vb{x}=\vb{v}\dd{t}$, convective/total derivative $\boxed{\frac{D\vb{v}}{D t}=\pdv{\vb{v}}{t}+\vb{v}\cdot\grad\vb{v}}$
    \item \textcolor{red}{Euler's equation} $\boxed{\rho\frac{D\vb{v}}{D t} = -\grad(P+\rho\phi_g) =-\grad P-\rho\grad\phi_g = -\grad P+\rho\vb{g}} $
    \item Streamlines, particle paths, streaklines
    \item Incompressible flow Bernoulli equation $\boxed{P+\frac{1}{2}\rho v^2+\rho\phi=C}$ along streamline
    \item Efflux coefficient is effective area/geometric area $<$ 1, Borda's mouthpiece 0.5
    \item Incompressible $\div\vb{v}=0$, irrotational vorticity $\vb{\omega\equiv\curl\vb{v}} = 0$, $\vb{v}=\grad\Phi$,$\laplacian\Phi=0$
    \item Circulation around a loop $\Gamma$ is $K=\oint_\Gamma\vb{v}\cdot\dd\vb{l}=\int_S(\curl\vb{v})\cdot\dd\vb{S}=\int\omega\cdot\dd\vb{S}$
    \item \begin{align*}\frac{D K}{D t}&=\oint_\Gamma\left(\frac{D\vb{v}}{D t}\cdot\dd\vb{l}+\vb{v}\cdot\frac{D(\dd\vb{l})}{Dt}\right) \\
                &= \oint_\Gamma\left(\grad(\frac{-P}{\rho}-\phi_g)\cdot\dd\vb{l}+\vb{v}\cdot\frac{D(\dd\vb{l})}{Dt}\right)\\
                &= \oint_\Gamma\left(\grad(\frac{-P}{\rho}-\phi_g)\cdot\dd\vb{l}+(\dd\vb{l}\cdot\grad)\vb{v}\right)\\
                &= \oint_\Gamma\left(\grad(\frac{-P}{\rho}-\phi_g)\cdot\dd\vb{l}+\grad(\frac{1}{2}v^2)\cdot\dd\vb{l}\right)\\
                &= \oint_\Gamma\grad\left(-\frac{P}{\rho}-\phi_g+\frac{1}{2}v^2\right)\dd\vb{l}\\
                &= 0
    \end{align*}because curl of gradient is 0
    \item $\boxed{\tau_{ij}=\eta\left(\pdv{v_i}{x_j}+\pdv{v_j}{x_i}\right)}=\eta\dv{2e_{ij}}{t}$ is the definition of viscosity $\eta$.
    \item Incompressible Navier-Stokes equation $\boxed{\rho\frac{D\vb{v}}{D t}=-\grad P+\rho\vb{g}+\eta\laplacian{\vb{v}}}$
    \item Poiseuille flow: steady state, $\frac{D\vb{v}}{Dt}=0$ ($\pdv{\vb{v}}[t]=0,\ \vb{v}\cdot\grad{\vb{v}}=0$), balance viscous shear force $\tau_{xy}\cdot\mathrm{area},\tau_{xy}=\eta\pdv{v_x}{y}$ ($v_y=0$ in these questions) and total force acted on this bulk of fluid (gravity/pressure difference ...), answer is parabolic velocity vs. distance
    \item Reynolds number $N_R=\frac{\rho v_0 d}{\eta}$ is a dimensionless number, where $\rho$ is density of fluid, $d$ is diameter of sphere/tube, $v_0$ is speed of sphere/average speed inside tube, $\eta$ is viscosity
    \item At high $N_R$ turbulence occurs. More viscous means lower $N_R$.
    \item Dipole (velocity) field (spherical BC): $\boxed{\Phi=v_0\cos\theta\left(r+\frac{a^3}{2r^2}\right)}$\begin{itemize}
            \item At \textcolor{red}{infinity, $\vb{v}=(v_0,0,0)$}, $\Phi=v_0x=v_0r\cos\theta$, $A_1=v_0$, $A_{i\neq 1}=0$
            \item $\Phi=\sum_{l=0}^\infty (A_l r^l+B_l r^{-l-1})P_l(\cos\theta)$
            \item At surface, $v_r=\dv{\Phi}{r}=0$, $v_0\cos\theta=\eval{(l+1)B_l r^{-l-2}P_l(\cos\theta)}_{r=a}$, $B_1=\frac{v_0a^3}{2}$
        \end{itemize}
\end{enumerate}

\input{shark.tex}
    
\end{document}